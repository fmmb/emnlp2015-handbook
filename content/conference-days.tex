
\chapter{Main Conference: Saturday, September 19}

\thispagestyle{emptyheader}


\section*{Overview}

\renewcommand{\arraystretch}{1.2}
\begin{SingleTrackSchedule}
  07:30 & -- & 18:00 &
  {\bfseries Registration} \hfill \emph{\RegistrationLoc}
  \\
  08:40 & -- & 10:00 &
  {\bfseries Session P1: Plenary Session} \hfill \emph{Main Auditorium}
  \\
 08:40 & -- & 09:00 & \textit{Opening Remarks and Introductory Speeches (General Chair, Program Co-Chairs and Local Co-Chairs)}\\
 09:00 & -- & 10:00 & \textit{Invited Talk: Deep Learning of Semantic Representations (Yoshua Bengio)}\\
  10:00 & -- & 10:30 &
  {\bfseries Coffee break} \hfill \emph{\CoffeeLoc}
  \\
  10:30 & -- & 12:10 &
  {\bfseries Session 1}\\

 & \multicolumn{3}{l}{%
 \begin{minipage}[t]{0.94\linewidth}
  \begin{tabular}{|>{\RaggedRight}p{0.235\linewidth}|>{\RaggedRight}p{0.235\linewidth}|>{\RaggedRight}p{0.235\linewidth}|>{\RaggedRight}p{0.235\linewidth}|}
  \hline
Semantics (Long + TACL Papers) & Machine Translation (Long + TACL Papers) & NLP for Web and Social Media, including Computational Social Science (Long Papers) & Long Paper Posters \rule{1\linewidth}{0.1pt} Short Paper Posters \\
\emph{\TrackALoc} & \emph{\TrackBLoc} & \emph{\TrackCLoc} & \emph{\TrackDLoc} \\
  \hline\end{tabular}
\end{minipage}
}\\
  12:10 & -- & 13:30 &
  {\bfseries Lunch} \hfill \emph{\LunchLoc}
  \\
  13:30 & -- & 15:10 &
  {\bfseries Session 2}\\

 & \multicolumn{3}{l}{%
 \begin{minipage}[t]{0.94\linewidth}
  \begin{tabular}{|>{\RaggedRight}p{0.235\linewidth}|>{\RaggedRight}p{0.235\linewidth}|>{\RaggedRight}p{0.235\linewidth}|>{\RaggedRight}p{0.235\linewidth}|}
  \hline
Statistical Models and Machine Learning Methods (Long + TACL Papers) & Tagging, Chunking and Parsing (Long +TACL Papers) & Summarization (Long Papers) & Long Paper Posters \rule{1\linewidth}{0.1pt} Short Paper Posters \\
\emph{\TrackALoc} & \emph{\TrackBLoc} & \emph{\TrackCLoc} & \emph{\TrackDLoc} \\
  \hline\end{tabular}
\end{minipage}
}\\
  15:10 & -- & 15:40 &
  {\bfseries Coffee break} \hfill \emph{\CoffeeLoc}
  \\
  15:40 & -- & 17:20 &
  {\bfseries Session 3}\\

 & \multicolumn{3}{l}{%
 \begin{minipage}[t]{0.94\linewidth}
  \begin{tabular}{|>{\RaggedRight}p{0.235\linewidth}|>{\RaggedRight}p{0.235\linewidth}|>{\RaggedRight}p{0.235\linewidth}|>{\RaggedRight}p{0.235\linewidth}|}
  \hline
Sentiment Analysis and Opinion Mining (Long Papers) & Semantics (Long +TACL Papers) & Information Retrieval and Question Answering (Long Papers) & Long + TACL Paper Posters \rule{1\linewidth}{0.1pt} Short Paper Posters \\
\emph{\TrackALoc} & \emph{\TrackBLoc} & \emph{\TrackCLoc} & \emph{\TrackDLoc} \\
  \hline\end{tabular}
\end{minipage}
}\\
\end{SingleTrackSchedule}


\clearpage{}


\section{Invited Speaker: Yoshua Bengio}

\index{Bengio, Yoshua}

\begin{center}
\textbf{\Large{}Deep Learning of Semantic Representations}{\Large{}\vspace{1em}
}
\par\end{center}{\Large \par}

\begin{center}
Saturday, September 19, 2015,  \vspace{1em}
\\
 \PlenaryLoc \\
 \vspace{1em}

\par\end{center}

\noindent \textbf{Abstract:} The core ingredient of deep learning
is the notion of distributed representation. This talk will start
by explaining its theoretical advantages, in comparison with non-parametric
methods based on counting frequencies of occurrence of observed tuples
of values (like with n-grams). The talk will then explain how having
multiple levels of representation, i.e., depth, can in principle give
another exponential advantage. Neural language models have been extremely
successful in recent years but extending their reach from language
modeling to machine translation is very appealing because it forces
the learned intermediate representations to capture meaning, and we
found that the resulting word embeddings are qualitatively different.
Recently, we introduced the notion of attention-based encoder-decoder
systems, with impressive results on machine translation several language
pairs and for mapping an image to a sentence, and these results will
conclude the talk.

\vspace{3em}


\vfill{}


\noindent \textbf{Biography:} Yoshua Bengio received a PhD in Computer
Science from McGill University, Canada in 1991. After two post-doctoral
years, one at M.I.T. with Michael Jordan and one at AT\&T Bell Laboratories
with Yann LeCun and Vladimir Vapnik, he became professor at the Department
of Computer Science and Operations Research at Université de Montréal.
He is the author of two books and more than 200 publications, the
most cited being in the areas of deep learning, recurrent neural networks,
probabilistic learning algorithms, natural language processing and
manifold learning. He is among the most cited Canadian computer scientists
and is or has been associate editor of the top journals in machine
learning and neural networks. Since '2000 he holds a Canada Research
Chair in Statistical Learning Algorithms, since '2006 an NSERC Industrial
Chair, since '2005 his is a Senior Fellow of the Canadian Institute
for Advanced Research and since 2014 he co-directs its program focused
on deep learning. He is on the board of the NIPS foundation and has
been program chair and general chair for NIPS. He has co-organized
the Learning Workshop for 14 years and co-created the new International
Conference on Learning Representations. His current interests are
centered around a quest for AI through machine learning, and include
fundamental questions on deep learning and representation learning,
the geometry of generalization in high-dimensional spaces, manifold
learning, biologically inspired learning algorithms, and challenging
applications of statistical machine learning.

\clearpage{}

\section[Session 1]{Session 1 Overview}
\begin{center}
\righthyphenmin2 \sloppy
\begin{tabular}{|p{0.33\columnwidth}|p{0.33\columnwidth}|p{0.33\columnwidth}|}
\hline
\bf Track A & \bf Track B & \bf Track C \\\hline
\it Semantics (Long + TACL Papers) & \it Machine Translation (Long + TACL Papers) & \it NLP for the Web and Social Media, including Computational Social Science (Long Papers) \\
\TrackALoc & \TrackBLoc & \TrackCLoc \\
\hline\hline
  \marginnote{\rotatebox{90}{10:30}}[2mm]
{}\papertableentry{papers-303} & {}\papertableentry{papers-444} & {}\papertableentry{papers-215}
  \\
  \hline
  \marginnote{\rotatebox{90}{10:55}}[2mm]
{}\papertableentry{papers-480} & {}\papertableentry{papers-501} & {}\papertableentry{papers-312}
  \\
  \hline
  \marginnote{\rotatebox{90}{11:20}}[2mm]
{}\papertableentry{papers-504} & {}\papertableentry{papers-630} & {}\papertableentry{papers-538}
  \\
  \hline
  \marginnote{\rotatebox{90}{11:45}}[2mm]
{[TACL] }\papertableentry{tacl-final-013} & {[TACL] }\papertableentry{tacl-final-003} & {}\papertableentry{papers-694}
  \\
\hline\end{tabular}\end{center}

\bigskip{}
\noindent \textbf{Track D:} \emph{Long Paper Posters} \hfill \emph{\sessionchair{Saif}{Mohammad}}\smallskip{}

\noindent {\PosterLoc \hfill \emph{10:30--12:10}}
\noindent \rule[0.5ex]{1\columnwidth}{1pt}
\begin{itemize}
\item []\textbf{Poster Cluster 1: Summarization (P1-4)}
\item \posterlistentry{papers-044}{}
\item \posterlistentry{papers-234}{}
\item \posterlistentry{papers-263}{}
\item \posterlistentry{papers-603}{}
\item []\textbf{Poster Cluster 2: Language and Vision (P5)}
\item \posterlistentry{papers-256}{}
\item []\textbf{Poster Cluster 3: Sentiment Analysis and Opinion Mining (P6-9)}
\item \posterlistentry{papers-089}{}
\item \posterlistentry{papers-152}{}
\item \posterlistentry{papers-617}{}
\item \posterlistentry{papers-646}{}
\end{itemize}

\bigskip{}
\noindent \textbf{Track E:} \emph{Short Paper Posters} \hfill \emph{\sessionchair{Kevin}{Duh}}\smallskip{}

\noindent {\PosterLoc \hfill \emph{10:30--12:10}}
\noindent \rule[0.5ex]{1\columnwidth}{1pt}
\begin{itemize}
\item []\textbf{Poster Cluster 1: Language and Vision (P1-3)}
\item \posterlistentry{papers-912}{}
\item \posterlistentry{papers-1268}{}
\item \posterlistentry{papers-1300}{}
\item []\textbf{Poster Cluster 2: Statistical Models and Machine Learning Methods (P4-16)}
\item \posterlistentry{papers-751}{}
\item \posterlistentry{papers-762}{}
\item \posterlistentry{papers-1048}{}
\item \posterlistentry{papers-1106}{}
\item \posterlistentry{papers-1125}{}
\item \posterlistentry{papers-1236}{}
\item \posterlistentry{papers-1246}{}
\item \posterlistentry{papers-1264}{}
\item \posterlistentry{papers-1342}{}
\item \posterlistentry{papers-1369}{}
\item \posterlistentry{papers-1381}{}
\item \posterlistentry{papers-1385}{}
\item \posterlistentry{papers-1457}{}
\end{itemize}

\clearpage

\newpage
\section*{Abstracts: Session 1}
\bigskip{}
\noindent{\bfseries\large Session 1A: Semantics (Long + TACL Papers)}\par
\noindent\TrackALoc\hfill\sessionchair{Yoav}{Artzi}\par
\bigskip{}
\paperabstract{papers-303}{10:30--10:55}{}
\paperabstract{papers-480}{10:55--11:20}{}
\paperabstract{papers-504}{11:20--11:45}{}
\paperabstract{tacl-final-013}{11:45--12:10}{[TACL] }
\clearpage
\noindent{\bfseries\large Session 1B: Machine Translation (Long + TACL Papers)}\par
\noindent\TrackBLoc\hfill\sessionchair{John}{DeNero}\par
\bigskip{}
\paperabstract{papers-444}{10:30--10:55}{}
\paperabstract{papers-501}{10:55--11:20}{}
\paperabstract{papers-630}{11:20--11:45}{}
\paperabstract{tacl-final-003}{11:45--12:10}{[TACL] }
\clearpage
\noindent{\bfseries\large Session 1C: NLP for the Web and Social Media, including Computational Social Science (Long Papers)}\par
\noindent\TrackCLoc\hfill\sessionchair{Brendan}{O'Connor}\par
\bigskip{}
\paperabstract{papers-215}{10:30--10:55}{}
\paperabstract{papers-312}{10:55--11:20}{}
\paperabstract{papers-538}{11:20--11:45}{}
\paperabstract{papers-694}{11:45--12:10}{}
\clearpage


\noindent{\bfseries\large Session 1D: Long Paper Posters} \hfill \emph{\sessionchair{Saif}{Mohammad}}\par
\noindent{\PosterLoc \hfill \emph{10:30--12:10}}\par
\bigskip{}
\posterabstract{papers-044}{}
\posterabstract{papers-234}{}
\posterabstract{papers-263}{}
\posterabstract{papers-603}{}
\posterabstract{papers-256}{}
\posterabstract{papers-089}{}
\posterabstract{papers-152}{}
\posterabstract{papers-617}{}
\posterabstract{papers-646}{}
\clearpage
\noindent{\bfseries\large Session 1E: Short Paper Posters} \hfill \emph{\sessionchair{Kevin}{Duh}}\par
\noindent{\PosterLoc \hfill \emph{10:30--12:10}}\par
\bigskip{}
\posterabstract{papers-912}{}
\posterabstract{papers-1268}{}
\posterabstract{papers-1300}{}
\posterabstract{papers-751}{}
\posterabstract{papers-762}{}
\posterabstract{papers-1048}{}
\posterabstract{papers-1106}{}
\posterabstract{papers-1125}{}
\posterabstract{papers-1236}{}
\posterabstract{papers-1246}{}
\posterabstract{papers-1264}{}
\posterabstract{papers-1342}{}
\posterabstract{papers-1369}{}
\posterabstract{papers-1381}{}
\posterabstract{papers-1385}{}
\posterabstract{papers-1457}{}
\clearpage

\section[Session 2]{Session 2 Overview}
\begin{center}
\righthyphenmin2 \sloppy
\begin{tabular}{|p{0.33\columnwidth}|p{0.33\columnwidth}|p{0.33\columnwidth}|}
\hline
\bf Track A & \bf Track B & \bf Track C \\\hline
\it Statistical Models and Machine Learning Methods (Long + TACL Papers) & \it Tagging, Chunking and Parsing (Long +TACL Papers) & \it Summarization (Long Papers) \\
\TrackALoc & \TrackBLoc & \TrackCLoc \\
\hline\hline
  \marginnote{\rotatebox{90}{13:30}}[2mm]
{}\papertableentry{papers-458} & {}\papertableentry{papers-054} & {}\papertableentry{papers-178}
  \\
  \hline
  \marginnote{\rotatebox{90}{13:55}}[2mm]
{}\papertableentry{papers-473} & {}\papertableentry{papers-162} & {}\papertableentry{papers-236}
  \\
  \hline
  \marginnote{\rotatebox{90}{14:20}}[2mm]
{}\papertableentry{papers-632} & {}\papertableentry{papers-524} & {}\papertableentry{papers-345}
  \\
  \hline
  \marginnote{\rotatebox{90}{14:45}}[2mm]
{[TACL] }\papertableentry{tacl-final-010} & {[TACL] }\papertableentry{tacl-final-016} & {}\papertableentry{papers-616}
  \\
\hline\end{tabular}\end{center}

\bigskip{}
\noindent \textbf{Track D:} \emph{Long Paper Posters} \hfill \emph{\sessionchair{Nate}{Chambers}}\smallskip{}

\noindent {\PosterLoc \hfill \emph{13:30--15:10}}
\noindent \rule[0.5ex]{1\columnwidth}{1pt}
\begin{itemize}
\item []\textbf{Poster Cluster 1: Text Mining and NLP Applications (P1-9)}
\item \posterlistentry{papers-102}{}
\item \posterlistentry{papers-259}{}
\item \posterlistentry{papers-432}{}
\item \posterlistentry{papers-445}{}
\item \posterlistentry{papers-475}{}
\item \posterlistentry{papers-507}{}
\item \posterlistentry{papers-544}{}
\item \posterlistentry{papers-608}{}
\item \posterlistentry{papers-614}{}
\end{itemize}

\bigskip{}
\noindent \textbf{Track E:} \emph{Short Paper Posters} \hfill \emph{\sessionchair{Wei}{Lu}}\smallskip{}

\noindent {\PosterLoc \hfill \emph{13:30--15:10}}
\noindent \rule[0.5ex]{1\columnwidth}{1pt}
\begin{itemize}
\item []\textbf{Poster Cluster 1: Information Extraction (P1-11)}
\item \posterlistentry{papers-821}{}
\item \posterlistentry{papers-902}{}
\item \posterlistentry{papers-995}{}
\item \posterlistentry{papers-1089}{}
\item \posterlistentry{papers-1183}{}
\item \posterlistentry{papers-1190}{}
\item \posterlistentry{papers-1267}{}
\item \posterlistentry{papers-1353}{}
\item \posterlistentry{papers-1383}{}
\item \posterlistentry{papers-1440}{}
\item \posterlistentry{papers-1464}{}
\item []\textbf{Poster Cluster 2: Information Retrieval and Question Answering (P12-16)}
\item \posterlistentry{papers-887}{}
\item \posterlistentry{papers-933}{}
\item \posterlistentry{papers-1249}{}
\item \posterlistentry{papers-1261}{}
\item \posterlistentry{papers-1389}{}
\end{itemize}

\clearpage

\newpage
\section*{Abstracts: Session 2}
\bigskip{}
\noindent{\bfseries\large Session 2A: Statistical Models and Machine Learning Methods (Long + TACL Papers)}\par
\noindent\TrackALoc\hfill\sessionchair{Alex}{Klementiev}\par
\bigskip{}
\paperabstract{papers-458}{13:30--13:55}{}
\paperabstract{papers-473}{13:55--14:20}{}
\paperabstract{papers-632}{14:20--14:45}{}
\paperabstract{tacl-final-010}{14:45--15:10}{[TACL] }
\clearpage
\noindent{\bfseries\large Session 2B: Tagging, Chunking and Parsing (Long +TACL Papers)}\par
\noindent\TrackBLoc\hfill\sessionchair{Eugene}{Charniak}\par
\bigskip{}
\paperabstract{papers-054}{13:30--13:55}{}
\paperabstract{papers-162}{13:55--14:20}{}
\paperabstract{papers-524}{14:20--14:45}{}
\paperabstract{tacl-final-016}{14:45--15:10}{[TACL] }
\clearpage
\noindent{\bfseries\large Session 2C: Summarization (Long Papers)}\par
\noindent\TrackCLoc\hfill\sessionchair{Yvette}{Graham}\par
\bigskip{}
\paperabstract{papers-178}{13:30--13:55}{}
\paperabstract{papers-236}{13:55--14:20}{}
\paperabstract{papers-345}{14:20--14:45}{}
\paperabstract{papers-616}{14:45--15:10}{}
\clearpage


\noindent{\bfseries\large Session 2D: Long Paper Posters} \hfill \emph{\sessionchair{Nate}{Chambers}}\par
\noindent{\PosterLoc \hfill \emph{13:30--15:10}}\par
\bigskip{}
\posterabstract{papers-102}{}
\posterabstract{papers-259}{}
\posterabstract{papers-432}{}
\posterabstract{papers-445}{}
\posterabstract{papers-475}{}
\posterabstract{papers-507}{}
\posterabstract{papers-544}{}
\posterabstract{papers-608}{}
\posterabstract{papers-614}{}
\clearpage
\noindent{\bfseries\large Session 2E: Short Paper Posters} \hfill \emph{\sessionchair{Wei}{Lu}}\par
\noindent{\PosterLoc \hfill \emph{13:30--15:10}}\par
\bigskip{}
\posterabstract{papers-821}{}
\posterabstract{papers-902}{}
\posterabstract{papers-995}{}
\posterabstract{papers-1089}{}
\posterabstract{papers-1183}{}
\posterabstract{papers-1190}{}
\posterabstract{papers-1267}{}
\posterabstract{papers-1353}{}
\posterabstract{papers-1383}{}
\posterabstract{papers-1440}{}
\posterabstract{papers-1464}{}
\posterabstract{papers-887}{}
\posterabstract{papers-933}{}
\posterabstract{papers-1249}{}
\posterabstract{papers-1261}{}
\posterabstract{papers-1389}{}
\clearpage

\section[Session 3]{Session 3 Overview}
\begin{center}
\righthyphenmin2 \sloppy
\begin{tabular}{|p{0.33\columnwidth}|p{0.33\columnwidth}|p{0.33\columnwidth}|}
\hline
\bf Track A & \bf Track B & \bf Track C \\\hline
\it Sentiment Analysis and Opinion Mining (Long Papers) & \it Semantics (Long +TACL Papers) & \it Information Retrieval and Question Answering (Long Papers) \\
\TrackALoc & \TrackBLoc & \TrackCLoc \\
\hline\hline
  \marginnote{\rotatebox{90}{15:40}}[2mm]
{}\papertableentry{papers-073} & {}\papertableentry{papers-536} & {}\papertableentry{papers-064}
  \\
  \hline
  \marginnote{\rotatebox{90}{16:05}}[2mm]
{}\papertableentry{papers-269} & {}\papertableentry{papers-635} & {}\papertableentry{papers-150}
  \\
  \hline
  \marginnote{\rotatebox{90}{16:30}}[2mm]
{}\papertableentry{papers-602} & {[TACL] }\papertableentry{tacl-final-001} & {}\papertableentry{papers-350}
  \\
  \hline
  \marginnote{\rotatebox{90}{16:55}}[2mm]
{}\papertableentry{papers-697} & {[TACL] }\papertableentry{tacl-final-018} & {}\papertableentry{papers-582}
  \\
\hline\end{tabular}\end{center}

\bigskip{}
\noindent \textbf{Track D:} \emph{Long + TACL Paper Posters} \hfill \emph{\sessionchair{Einat}{Minkov}}\smallskip{}

\noindent {\PosterLoc \hfill \emph{15:40--17:20}}
\noindent \rule[0.5ex]{1\columnwidth}{1pt}
\begin{itemize}
\item []\textbf{Poster Cluster 1: Information Extraction (P1-9)}
\item \posterlistentry{papers-221}{}
\item \posterlistentry{papers-232}{}
\item \posterlistentry{papers-385}{}
\item \posterlistentry{papers-453}{}
\item \posterlistentry{papers-474}{}
\item \posterlistentry{papers-526}{}
\item \posterlistentry{papers-708}{}
\item \posterlistentry{tacl-final-006}{[TACL] }
\item \posterlistentry{tacl-final-008}{[TACL] }
\end{itemize}

\bigskip{}
\noindent \textbf{Track E:} \emph{Short Paper Posters} \hfill \emph{\sessionchair{Wei}{Xu}}\smallskip{}

\noindent {\PosterLoc \hfill \emph{15:40--17:20}}
\noindent \rule[0.5ex]{1\columnwidth}{1pt}
\begin{itemize}
\item []\textbf{Poster Cluster 1: Text Mining and NLP Applications (P1-13)}
\item \posterlistentry{papers-1393}{}
\item \posterlistentry{papers-785}{}
\item \posterlistentry{papers-850}{}
\item \posterlistentry{papers-774}{}
\item \posterlistentry{papers-1062}{}
\item \posterlistentry{papers-1016}{}
\item \posterlistentry{papers-1120}{}
\item \posterlistentry{papers-1245}{}
\item \posterlistentry{papers-1156}{}
\item \posterlistentry{papers-1192}{}
\item \posterlistentry{papers-1336}{}
\item \posterlistentry{papers-776}{}
\item \posterlistentry{papers-777}{}
\end{itemize}

\clearpage

\newpage
\section*{Abstracts: Session 3}
\bigskip{}
\noindent{\bfseries\large Session 3A: Sentiment Analysis and Opinion Mining (Long Papers)}\par
\noindent\TrackALoc\hfill\sessionchair{Janyce}{Wiebe}\par
\bigskip{}
\paperabstract{papers-073}{15:40--16:05}{}
\paperabstract{papers-269}{16:05--16:30}{}
\paperabstract{papers-602}{16:30--16:55}{}
\paperabstract{papers-697}{16:55--17:20}{}
\clearpage
\noindent{\bfseries\large Session 3B: Semantics (Long +TACL Papers)}\par
\noindent\TrackBLoc\hfill\sessionchair{Regina}{Barzilay}\par
\bigskip{}
\paperabstract{papers-536}{15:40--16:05}{}
\paperabstract{papers-635}{16:05--16:30}{}
\paperabstract{tacl-final-001}{16:30--16:55}{[TACL] }
\paperabstract{tacl-final-018}{16:55--17:20}{[TACL] }
\clearpage
\noindent{\bfseries\large Session 3C: Information Retrieval and Question Answering (Long Papers)}\par
\noindent\TrackCLoc\hfill\sessionchair{Scott}{Wen-tau Yih}\par
\bigskip{}
\paperabstract{papers-064}{15:40--16:05}{}
\paperabstract{papers-150}{16:05--16:30}{}
\paperabstract{papers-350}{16:30--16:55}{}
\paperabstract{papers-582}{16:55--17:20}{}
\clearpage


\noindent{\bfseries\large Session 3D: Long + TACL Paper Posters} \hfill \emph{\sessionchair{Einat}{Minkov}}\par
\noindent{\PosterLoc \hfill \emph{15:40--17:20}}\par
\bigskip{}
\posterabstract{papers-221}{}
\posterabstract{papers-232}{}
\posterabstract{papers-385}{}
\posterabstract{papers-453}{}
\posterabstract{papers-474}{}
\posterabstract{papers-526}{}
\posterabstract{papers-708}{}
\posterabstract{tacl-final-006}{[TACL] }
\posterabstract{tacl-final-008}{[TACL] }
\clearpage
\noindent{\bfseries\large Session 3E: Short Paper Posters} \hfill \emph{\sessionchair{Wei}{Xu}}\par
\noindent{\PosterLoc \hfill \emph{15:40--17:20}}\par
\bigskip{}
\posterabstract{papers-1393}{}
\posterabstract{papers-785}{}
\posterabstract{papers-850}{}
\posterabstract{papers-774}{}
\posterabstract{papers-1062}{}
\posterabstract{papers-1016}{}
\posterabstract{papers-1120}{}
\posterabstract{papers-1245}{}
\posterabstract{papers-1156}{}
\posterabstract{papers-1192}{}
\posterabstract{papers-1336}{}
\posterabstract{papers-776}{}
\posterabstract{papers-777}{}
\clearpage



\chapter{Main Conference: Sunday, September 20}


\section*{Overview}

\renewcommand{\arraystretch}{1.2}
\begin{SingleTrackSchedule}
  07:30 & -- & 18:00 &
  {\bfseries Registration} \hfill \emph{\RegistrationLoc}
  \\
  08:00 & -- & 09:00 &
  {\bfseries Morning Coffee} \hfill \emph{\MorningLoc}
  \\
  09:00 & -- & 10:00 &
  {\bfseries Session P2: Plenary Session} \hfill \emph{Main Auditorium}
  \\
 & & & \textit{Invited Talk: Measuring How Elected Officials and Constituents Communicate (Justin Grimmer)}\\
  10:00 & -- & 10:30 &
  {\bfseries Coffee break} \hfill \emph{\CoffeeLoc}
  \\
  10:30 & -- & 12:10 &
  {\bfseries Session 4}\\

 & \multicolumn{3}{l}{%
 \begin{minipage}[t]{0.94\linewidth}
  \begin{tabular}{|>{\RaggedRight}p{0.235\linewidth}|>{\RaggedRight}p{0.235\linewidth}|>{\RaggedRight}p{0.235\linewidth}|>{\RaggedRight}p{0.235\linewidth}|}
  \hline
Information Extraction (Long Papers) & Statistical Models and Machine Learning Methods (Long Papers) & Discourse (Long +TACL Papers) & Long Paper Posters \rule{1\linewidth}{0.1pt} Short Paper Posters \\
\emph{\TrackALoc} & \emph{\TrackBLoc} & \emph{\TrackCLoc} & \emph{\TrackDLoc} \\
  \hline\end{tabular}
\end{minipage}
}\\
  12:10 & -- & 12:50 &
  {\bfseries Lunch} \hfill \emph{\LunchLoc}
  \\
  12:50 & -- & 13:30 &
  {\bfseries Session P3: SIGDAT business meeting} \hfill \emph{Main Auditorium}
  \\
  13:30 & -- & 15:10 &
  {\bfseries Session 5}\\

 & \multicolumn{3}{l}{%
 \begin{minipage}[t]{0.94\linewidth}
  \begin{tabular}{|>{\RaggedRight}p{0.235\linewidth}|>{\RaggedRight}p{0.235\linewidth}|>{\RaggedRight}p{0.235\linewidth}|>{\RaggedRight}p{0.235\linewidth}|}
  \hline
Text Mining and NLP Applications (Long + TACL Papers) & Semantics (Long +TACL Papers) & Phonology and Word Segmentation (Long Papers) & Long Paper Posters \rule{1\linewidth}{0.1pt} Short Paper Posters \\
\emph{\TrackALoc} & \emph{\TrackBLoc} & \emph{\TrackCLoc} & \emph{\TrackDLoc} \\
  \hline\end{tabular}
\end{minipage}
}\\
  15:10 & -- & 15:40 &
  {\bfseries Coffee break} \hfill \emph{\CoffeeLoc}
  \\
  15:40 & -- & 17:20 &
  {\bfseries Session 6}\\

 & \multicolumn{3}{l}{%
 \begin{minipage}[t]{0.94\linewidth}
  \begin{tabular}{|>{\RaggedRight}p{0.235\linewidth}|>{\RaggedRight}p{0.235\linewidth}|>{\RaggedRight}p{0.235\linewidth}|>{\RaggedRight}p{0.235\linewidth}|}
  \hline
Machine Translation (Long Papers) & Sentiment Analysis and Opinion Mining / Tagging, Chunking and Parsing (Long Papers) & Language and Vision / Information Extraction (Long Papers) & Long Paper Posters \rule{1\linewidth}{0.1pt} Short Paper Posters \\
\emph{\TrackALoc} & \emph{\TrackBLoc} & \emph{\TrackCLoc} & \emph{\TrackDLoc} \\
  \hline\end{tabular}
\end{minipage}
}\\
  19:00 & -- & 23:00 &
  {\bfseries Conference Dinner} \hfill \emph{Pateo Alfacinha}
  \\
\end{SingleTrackSchedule}


\clearpage{}


\section{Invited Speaker: Justin Grimmer}

\index{Bengio, Yoshua}

\begin{center}
\textbf{\Large{}Measuring How Elected Officials and Constituents Communicate}{\Large{}\vspace{1em}
}
\par\end{center}{\Large \par}

\begin{center}
Sunday, September 20, 2015,  \vspace{1em}
\\
 \PlenaryLoc \\
 \vspace{1em}

\par\end{center}

\noindent \textbf{Abstract:} This talk will show how elected officials
use communication to cultivate support with constituents, how constituents
express their views to elected officials, and why biases in both kinds
of communication matter for political representation. To demonstrate
the bias and its effects, I propose to use novel collections of political
texts and new text as data methods. Using the new data and methods,
I will show how the incentives of communication contribute to perceptions
of an angry public and vitriolic politicians. Among elected officials,
the ideologically extreme members of Congress disproportionately participate
in policy debates, resulting in political debates that occur between
the most extreme members of each party. Among constituents, the most
ideologically extreme and angry voters disproportionately contact
their member of Congress, creating the impression of a polarized and
vitriolic public. The talk will explain how the findings help us to
understand how representation occurs in American politics, while also
explaining how computational tools can help address questions in the
social sciences.

\vspace{3em}


\vfill{}


\noindent \textbf{Biography:} Justin Grimmer is an associate professor
of political science at Stanford University. His research examines
how representation occurs in American politics using new statistical
methods. His first book Representational Style in Congress: What Legislators
Say and Why It Matters (Cambridge University Press, 2013) shows how
senators define the type of representation they provide constituents
and how this affects constituents' evaluations and was awarded the
2014 Richard Fenno Prize. His second book The Impression of Influence:
How Legislator Communication and Government Spending Cultivate a Personal
Vote (Princeton University Press, 2014 with Sean J. Westwood and Solomon
Messing) demonstrates how legislators ensure they receive credit for
government actions. His work has appeared in the American Political
Science Review, American Journal of Political Science, Journal of
Politics, Political Analysis, Proceedings of the National Academy
of Sciences, Regulation and Governance, and Poetics.

\clearpage{}

\section[Session 4]{Session 4 Overview}
\begin{center}
\righthyphenmin2 \sloppy
\begin{tabular}{|p{0.33\columnwidth}|p{0.33\columnwidth}|p{0.33\columnwidth}|}
\hline
\bf Track A & \bf Track B & \bf Track C \\\hline
\it Information Extraction (Long Papers) & \it Statistical Models and Machine Learning Methods (Long Papers) & \it Discourse (Long +TACL Papers) \\
\TrackALoc & \TrackBLoc & \TrackCLoc \\
\hline\hline
  \marginnote{\rotatebox{90}{10:30}}[2mm]
{}\papertableentry{papers-080} & {}\papertableentry{papers-119} & {}\papertableentry{papers-417}
  \\
  \hline
  \marginnote{\rotatebox{90}{10:55}}[2mm]
{}\papertableentry{papers-352} & {}\papertableentry{papers-456} & {}\papertableentry{papers-568}
  \\
  \hline
  \marginnote{\rotatebox{90}{11:20}}[2mm]
{}\papertableentry{papers-595} & {}\papertableentry{papers-547} & {[TACL] }\papertableentry{tacl-final-004}
  \\
  \hline
  \marginnote{\rotatebox{90}{11:45}}[2mm]
{}\papertableentry{papers-707} & {}\papertableentry{papers-676} & {[TACL] }\papertableentry{tacl-final-012}
  \\
\hline\end{tabular}\end{center}

\bigskip{}
\noindent \textbf{Track D:} \emph{Long Paper Posters} \hfill \emph{\sessionchair{Sam}{Bowman}}\smallskip{}

\noindent {\PosterLoc \hfill \emph{10:30--12:10}}
\noindent \rule[0.5ex]{1\columnwidth}{1pt}
\begin{itemize}
\item []\textbf{Poster Cluster 1: Semantics (P1-8)}
\item \posterlistentry{papers-318}{}
\item \posterlistentry{papers-455}{}
\item \posterlistentry{papers-520}{}
\item \posterlistentry{papers-559}{}
\item \posterlistentry{papers-597}{}
\item \posterlistentry{papers-606}{}
\item \posterlistentry{papers-653}{}
\item \posterlistentry{papers-654}{}
\end{itemize}

\bigskip{}
\noindent \textbf{Track E:} \emph{Short Paper Posters} \hfill \emph{\sessionchair{Maja}{Popovic}}\smallskip{}

\noindent {\PosterLoc \hfill \emph{10:30--12:10}}
\noindent \rule[0.5ex]{1\columnwidth}{1pt}
\begin{itemize}
\item []\textbf{Poster Cluster 1: Machine Translation and Multilinguality (P1-13)}
\item \posterlistentry{papers-804}{}
\item \posterlistentry{papers-950}{}
\item \posterlistentry{papers-1058}{}
\item \posterlistentry{papers-1061}{}
\item \posterlistentry{papers-1078}{}
\item \posterlistentry{papers-1127}{}
\item \posterlistentry{papers-1222}{}
\item \posterlistentry{papers-1235}{}
\item \posterlistentry{papers-1275}{}
\item \posterlistentry{papers-1359}{}
\item \posterlistentry{papers-1451}{}
\item \posterlistentry{papers-1455}{}
\item \posterlistentry{papers-1470}{}
\item []\textbf{Poster Cluster 2: Computational Psycholinguistics (P14-16)}
\item \posterlistentry{papers-1274}{}
\item \posterlistentry{papers-1314}{}
\item \posterlistentry{papers-1323}{}
\end{itemize}

\clearpage

\newpage
\section*{Abstracts: Session 4}
\bigskip{}
\noindent{\bfseries\large Session 4A: Information Extraction (Long Papers)}\par
\noindent\TrackALoc\hfill\sessionchair{Heng}{Ji}\par
\bigskip{}
\paperabstract{papers-080}{10:30--10:55}{}
\paperabstract{papers-352}{10:55--11:20}{}
\paperabstract{papers-595}{11:20--11:45}{}
\paperabstract{papers-707}{11:45--12:10}{}
\clearpage
\noindent{\bfseries\large Session 4B: Statistical Models and Machine Learning Methods (Long Papers)}\par
\noindent\TrackBLoc\hfill\sessionchair{Chris}{Dyer}\par
\bigskip{}
\paperabstract{papers-119}{10:30--10:55}{}
\paperabstract{papers-456}{10:55--11:20}{}
\paperabstract{papers-547}{11:20--11:45}{}
\paperabstract{papers-676}{11:45--12:10}{}
\clearpage
\noindent{\bfseries\large Session 4C: Discourse (Long +TACL Papers)}\par
\noindent\TrackCLoc\hfill\sessionchair{Pascal}{Denis}\par
\bigskip{}
\paperabstract{papers-417}{10:30--10:55}{}
\paperabstract{papers-568}{10:55--11:20}{}
\paperabstract{tacl-final-004}{11:20--11:45}{[TACL] }
\paperabstract{tacl-final-012}{11:45--12:10}{[TACL] }
\clearpage


\noindent{\bfseries\large Session 4D: Long Paper Posters} \hfill \emph{\sessionchair{Sam}{Bowman}}\par
\noindent{\PosterLoc \hfill \emph{10:30--12:10}}\par
\bigskip{}
\posterabstract{papers-318}{}
\posterabstract{papers-455}{}
\posterabstract{papers-520}{}
\posterabstract{papers-559}{}
\posterabstract{papers-597}{}
\posterabstract{papers-606}{}
\posterabstract{papers-653}{}
\posterabstract{papers-654}{}
\clearpage
\noindent{\bfseries\large Session 4E: Short Paper Posters} \hfill \emph{\sessionchair{Maja}{Popovic}}\par
\noindent{\PosterLoc \hfill \emph{10:30--12:10}}\par
\bigskip{}
\posterabstract{papers-804}{}
\posterabstract{papers-950}{}
\posterabstract{papers-1058}{}
\posterabstract{papers-1061}{}
\posterabstract{papers-1078}{}
\posterabstract{papers-1127}{}
\posterabstract{papers-1222}{}
\posterabstract{papers-1235}{}
\posterabstract{papers-1275}{}
\posterabstract{papers-1359}{}
\posterabstract{papers-1451}{}
\posterabstract{papers-1455}{}
\posterabstract{papers-1470}{}
\posterabstract{papers-1274}{}
\posterabstract{papers-1314}{}
\posterabstract{papers-1323}{}
\clearpage

\section[Session 5]{Session 5 Overview}
\begin{center}
\righthyphenmin2 \sloppy
\begin{tabular}{|p{0.33\columnwidth}|p{0.33\columnwidth}|p{0.33\columnwidth}|}
\hline
\bf Track A & \bf Track B & \bf Track C \\\hline
\it Text Mining and NLP Applications (Long + TACL Papers) & \it Semantics (Long +TACL Papers) & \it Phonology and Word Segmentation (Long Papers) \\
\TrackALoc & \TrackBLoc & \TrackCLoc \\
\hline\hline
  \marginnote{\rotatebox{90}{13:30}}[2mm]
{[TACL] }\papertableentry{tacl-final-014} & {}\papertableentry{papers-222} & {}\papertableentry{papers-164}
  \\
  \hline
  \marginnote{\rotatebox{90}{13:55}}[2mm]
{}\papertableentry{papers-658} & {}\papertableentry{papers-564} & {}\papertableentry{papers-336}
  \\
  \hline
  \marginnote{\rotatebox{90}{14:20}}[2mm]
{[TACL] }\papertableentry{tacl-final-002} & {}\papertableentry{papers-615} & {}\papertableentry{papers-470}
  \\
  \hline
  \marginnote{\rotatebox{90}{14:45}}[2mm]
{[TACL] }\papertableentry{tacl-final-007} & {[TACL] }\papertableentry{tacl-final-017} & {}\papertableentry{papers-681}
  \\
\hline\end{tabular}\end{center}

\bigskip{}
\noindent \textbf{Track D:} \emph{Long Paper Posters} \hfill \emph{\sessionchair{Boxing}{Chen}}\smallskip{}

\noindent {\PosterLoc \hfill \emph{13:30--15:10}}
\noindent \rule[0.5ex]{1\columnwidth}{1pt}
\begin{itemize}
\item []\textbf{Poster Cluster 1: Machine Translation and Multilinguality (P1-8)}
\item \posterlistentry{papers-328}{}
\item \posterlistentry{papers-354}{}
\item \posterlistentry{papers-381}{}
\item \posterlistentry{papers-383}{}
\item \posterlistentry{papers-405}{}
\item \posterlistentry{papers-477}{}
\item \posterlistentry{papers-663}{}
\item \posterlistentry{papers-685}{}
\end{itemize}

\bigskip{}
\noindent \textbf{Track E:} \emph{Short Paper Posters} \hfill \emph{\sessionchair{Stephen}{Clark}}\smallskip{}

\noindent {\PosterLoc \hfill \emph{13:30--15:10}}
\noindent \rule[0.5ex]{1\columnwidth}{1pt}
\begin{itemize}
\item []\textbf{Poster Cluster 1: Tagging, Syntax and Parsing (P1-12)}
\item \posterlistentry{papers-794}{}
\item \posterlistentry{papers-837}{}
\item \posterlistentry{papers-839}{}
\item \posterlistentry{papers-895}{}
\item \posterlistentry{papers-968}{}
\item \posterlistentry{papers-1001}{}
\item \posterlistentry{papers-1238}{}
\item \posterlistentry{papers-1241}{}
\item \posterlistentry{papers-1301}{}
\item \posterlistentry{papers-1316}{}
\item \posterlistentry{papers-1368}{}
\item \posterlistentry{papers-1433}{}
\end{itemize}

\clearpage

\newpage
\section*{Abstracts: Parallel Session 5}
\bigskip{}
\noindent{\bfseries\large Session 5A: Text Mining and NLP Applications (Long + TACL Papers)}\par
\noindent\TrackALoc\hfill\sessionchair{Marie-Francine}{Moens}\par
\bigskip{}
\paperabstract{tacl-final-014}{13:30--13:55}{[TACL] }
\paperabstract{papers-658}{13:55--14:20}{}
\paperabstract{tacl-final-002}{14:20--14:45}{[TACL] }
\paperabstract{tacl-final-007}{14:45--15:10}{[TACL] }
\clearpage
\noindent{\bfseries\large Session 5B: Semantics (Long +TACL Papers)}\par
\noindent\TrackBLoc\hfill\sessionchair{Benjamin}{Van Durme}\par
\bigskip{}
\paperabstract{papers-222}{13:30--13:55}{}
\paperabstract{papers-564}{13:55--14:20}{}
\paperabstract{papers-615}{14:20--14:45}{}
\paperabstract{tacl-final-017}{14:45--15:10}{[TACL] }
\clearpage
\noindent{\bfseries\large Session 5C: Phonology and Word Segmentation (Long Papers)}\par
\noindent\TrackCLoc\hfill\sessionchair{Yue}{Zhang}\par
\bigskip{}
\paperabstract{papers-164}{13:30--13:55}{}
\paperabstract{papers-336}{13:55--14:20}{}
\paperabstract{papers-470}{14:20--14:45}{}
\paperabstract{papers-681}{14:45--15:10}{}
\clearpage


\noindent{\bfseries\large Session 5D: Long Paper Posters} \hfill \emph{\sessionchair{Boxing}{Chen}}\par
\noindent{\PosterLoc \hfill \emph{13:30--15:10}}\par
\bigskip{}
\posterabstract{papers-328}{}
\posterabstract{papers-354}{}
\posterabstract{papers-381}{}
\posterabstract{papers-383}{}
\posterabstract{papers-405}{}
\posterabstract{papers-477}{}
\posterabstract{papers-663}{}
\posterabstract{papers-685}{}
\clearpage
\noindent{\bfseries\large Session 5E: Short Paper Posters} \hfill \emph{\sessionchair{Stephen}{Clark}}\par
\noindent{\PosterLoc \hfill \emph{13:30--15:10}}\par
\bigskip{}
\posterabstract{papers-794}{}
\posterabstract{papers-837}{}
\posterabstract{papers-839}{}
\posterabstract{papers-895}{}
\posterabstract{papers-968}{}
\posterabstract{papers-1001}{}
\posterabstract{papers-1238}{}
\posterabstract{papers-1241}{}
\posterabstract{papers-1301}{}
\posterabstract{papers-1316}{}
\posterabstract{papers-1368}{}
\posterabstract{papers-1433}{}
\clearpage

\section[Session 6]{Session 6 Overview}
\begin{center}
\righthyphenmin2 \sloppy
\begin{tabular}{|p{0.33\columnwidth}|p{0.33\columnwidth}|p{0.33\columnwidth}|}
\hline
\bf Track A & \bf Track B & \bf Track C \\\hline
\it Machine Translation (Long Papers) & \it Sentiment Analysis and Opinion Mining / Tagging, Chunking and Parsing (Long Papers) & \it Language and Vision / Information Extraction (Long Papers) \\
\TrackALoc & \TrackBLoc & \TrackCLoc \\
\hline\hline
  \marginnote{\rotatebox{90}{15:40}}[2mm]
{}\papertableentry{papers-216} & {}\papertableentry{papers-011} & {}\papertableentry{papers-349}
  \\
  \hline
  \marginnote{\rotatebox{90}{16:05}}[2mm]
{}\papertableentry{papers-250} & {}\papertableentry{papers-651} & {}\papertableentry{papers-494}
  \\
  \hline
  \marginnote{\rotatebox{90}{16:30}}[2mm]
{}\papertableentry{papers-438} & {}\papertableentry{papers-465} & {}\papertableentry{papers-138}
  \\
  \hline
  \marginnote{\rotatebox{90}{16:55}}[2mm]
{}\papertableentry{papers-672} & {}\papertableentry{papers-636} & {}\papertableentry{papers-674}
  \\
\hline\end{tabular}\end{center}

\bigskip{}
\noindent \textbf{Track D:} \emph{Long Paper Posters} \hfill \emph{\sessionchair{Noah}{Smith}}\smallskip{}

\noindent {\PosterLoc \hfill \emph{15:40--17:20}}
\noindent \rule[0.5ex]{1\columnwidth}{1pt}
\begin{itemize}
\item []\textbf{Poster Cluster 1: Statistical Models and Machine Learning Methods (P1-11)}
\item \posterlistentry{papers-223}{}
\item \posterlistentry{papers-261}{}
\item \posterlistentry{papers-072}{}
\item \posterlistentry{papers-414}{}
\item \posterlistentry{papers-508}{}
\item \posterlistentry{papers-509}{}
\item \posterlistentry{papers-570}{}
\item \posterlistentry{papers-626}{}
\item \posterlistentry{papers-684}{}
\item \posterlistentry{papers-715}{}
\item \posterlistentry{papers-717}{}
\end{itemize}

\bigskip{}
\noindent \textbf{Track E:} \emph{Short Paper Posters} \hfill \emph{\sessionchair{Ido}{Dagan}}\smallskip{}

\noindent {\PosterLoc \hfill \emph{15:40--17:20}}
\noindent \rule[0.5ex]{1\columnwidth}{1pt}
\begin{itemize}
\item []\textbf{Poster Cluster 1: Semantics (P1-13)}
\item \posterlistentry{papers-827}{}
\item \posterlistentry{papers-865}{}
\item \posterlistentry{papers-872}{}
\item \posterlistentry{papers-920}{}
\item \posterlistentry{papers-921}{}
\item \posterlistentry{papers-926}{}
\item \posterlistentry{papers-945}{}
\item \posterlistentry{papers-1152}{}
\item \posterlistentry{papers-1184}{}
\item \posterlistentry{papers-1191}{}
\item \posterlistentry{papers-1304}{}
\item \posterlistentry{papers-1333}{}
\item \posterlistentry{papers-1394}{}
\end{itemize}

\clearpage

\newpage
\section*{Abstracts: Session 6}
\bigskip{}
\noindent{\bfseries\large Session 6A: Machine Translation (Long Papers)}\par
\noindent\TrackALoc\hfill\sessionchair{Lucia}{Specia}\par
\bigskip{}
\paperabstract{papers-216}{15:40--16:05}{}
\paperabstract{papers-250}{16:05--16:30}{}
\paperabstract{papers-438}{16:30--16:55}{}
\paperabstract{papers-672}{16:55--17:20}{}
\clearpage
\noindent{\bfseries\large Session 6B: Sentiment Analysis and Opinion Mining / Tagging, Chunking and Parsing (Long Papers)}\par
\noindent\TrackBLoc\hfill\sessionchair{Bing}{Liu and Dan Bikel}\par
\bigskip{}
\paperabstract{papers-011}{15:40--16:05}{}
\paperabstract{papers-651}{16:05--16:30}{}
\paperabstract{papers-465}{16:30--16:55}{}
\paperabstract{papers-636}{16:55--17:20}{}
\clearpage
\noindent{\bfseries\large Session 6C: Language and Vision / Information Extraction (Long Papers)}\par
\noindent\TrackCLoc\hfill\sessionchair{Meg}{Mitchell and Partha Talukdar}\par
\bigskip{}
\paperabstract{papers-349}{15:40--16:05}{}
\paperabstract{papers-494}{16:05--16:30}{}
\paperabstract{papers-138}{16:30--16:55}{}
\paperabstract{papers-674}{16:55--17:20}{}
\clearpage


\noindent{\bfseries\large Session 6D: Long Paper Posters} \hfill \emph{\sessionchair{Noah}{Smith}}\par
\noindent{\PosterLoc \hfill \emph{15:40--17:20}}\par
\bigskip{}
\posterabstract{papers-223}{}
\posterabstract{papers-261}{}
\posterabstract{papers-414}{}
\posterabstract{papers-508}{}
\posterabstract{papers-509}{}
\posterabstract{papers-570}{}
\posterabstract{papers-626}{}
\posterabstract{papers-684}{}
\posterabstract{papers-715}{}
\posterabstract{papers-717}{}
\clearpage
\noindent{\bfseries\large Session 6E: Short Paper Posters} \hfill \emph{\sessionchair{Ido}{Dagan}}\par
\noindent{\PosterLoc \hfill \emph{15:40--17:20}}\par
\bigskip{}
\posterabstract{papers-827}{}
\posterabstract{papers-865}{}
\posterabstract{papers-872}{}
\posterabstract{papers-920}{}
\posterabstract{papers-921}{}
\posterabstract{papers-926}{}
\posterabstract{papers-945}{}
\posterabstract{papers-1152}{}
\posterabstract{papers-1184}{}
\posterabstract{papers-1191}{}
\posterabstract{papers-1304}{}
\posterabstract{papers-1333}{}
\posterabstract{papers-1394}{}
\clearpage



\chapter{Main Conference: Monday, September 21}


\section*{Overview}

\renewcommand{\arraystretch}{1.2}
\begin{SingleTrackSchedule}
  07:30 & -- & 18:00 &
  {\bfseries Registration} \hfill \emph{\RegistrationLoc}
  \\
  08:00 & -- & 09:00 &
  {\bfseries Morning Coffee} \hfill \emph{\MorningLoc}
  \\
  09:00 & -- & 10:00 &
  {\bfseries Session P4: Plenary Session} \hfill \emph{Main Auditorium}
  \\
 09:00 & -- & 09:05 & \textit{Best Paper Awards (Chris Callison-Burch and Jian Su)}\\
 09:05 & -- & 09:30 & \paperlistentry{papers-662}\\
 09:30 & -- & 09:55 & \paperlistentry{papers-072}\\
 09:55 & -- & 10:05 & \paperlistentry{papers-536}\\
  10:05 & -- & 10:30 &
  {\bfseries Coffee break} \hfill \emph{\CoffeeLoc}
  \\
  10:30 & -- & 12:10 &
  {\bfseries Session 7}\\

 & \multicolumn{3}{l}{%
 \begin{minipage}[t]{0.94\linewidth}
  \begin{tabular}{|>{\RaggedRight}p{0.235\linewidth}|>{\RaggedRight}p{0.235\linewidth}|>{\RaggedRight}p{0.235\linewidth}|>{\RaggedRight}p{0.235\linewidth}|}
  \hline
Semantics (Long +TACL Papers) & Information Extraction (Long Papers) & Computational Psycholinguistics / Machine Translation (Long Papers) & Long +TACL Paper Posters \rule{1\linewidth}{0.1pt} Short Paper Posters \\
\emph{\TrackALoc} & \emph{\TrackBLoc} & \emph{\TrackCLoc} & \emph{\TrackDLoc} \\
  \hline\end{tabular}
\end{minipage}
}\\
  12:10 & -- & 13:30 &
  {\bfseries Lunch} \hfill \emph{\LunchLoc}
  \\
  13:30 & -- & 15:15 &
  {\bfseries Session 8}\\

 & \multicolumn{3}{l}{%
 \begin{minipage}[t]{0.94\linewidth}
  \begin{tabular}{|>{\RaggedRight}p{0.235\linewidth}|>{\RaggedRight}p{0.235\linewidth}|>{\RaggedRight}p{0.235\linewidth}|>{\RaggedRight}p{0.235\linewidth}|}
  \hline
Fun and Quirky Topics (Short Papers) & Semantics (Short Papers) & Statistical Models and Machine Learning Methods / Machine Translation (Short Papers) & Long Paper Posters \rule{1\linewidth}{0.1pt} Short Paper Posters \\
\emph{\TrackALoc} & \emph{\TrackBLoc} & \emph{\TrackCLoc} & \emph{\TrackDLoc} \\
  \hline\end{tabular}
\end{minipage}
}\\
  15:15 & -- & 15:40 &
  {\bfseries Coffee break} \hfill \emph{\CoffeeLoc}
  \\
  15:40 & -- & 17:20 &
  {\bfseries Session 9}\\

 & \multicolumn{3}{l}{%
 \begin{minipage}[t]{0.94\linewidth}
  \begin{tabular}{|>{\RaggedRight}p{0.235\linewidth}|>{\RaggedRight}p{0.235\linewidth}|>{\RaggedRight}p{0.235\linewidth}|>{\RaggedRight}p{0.235\linewidth}|}
  \hline
Statistical Models and Machine Learning Methods (Long + TACL Papers) & Text Mining and NLP Applications (Long Papers) & Spoken Language Processing and Language Modeling (Long Papers) & Long Paper Posters \rule{1\linewidth}{0.1pt} Short Paper Posters \\
\emph{\TrackALoc} & \emph{\TrackBLoc} & \emph{\TrackCLoc} & \emph{\TrackDLoc} \\
  \hline\end{tabular}
\end{minipage}
}\\
  17:30 & -- & 17:50 &
  {\bfseries Session P5: Closing Remarks} \hfill \emph{Main Auditorium}
  \\
  18:30 & -- & 20:00 &
  {\bfseries Farewell Drink} \hfill \emph{Culturgest}
  \\
\end{SingleTrackSchedule}


\clearpage{}

\section[Session 7]{Session 7 Overview}
\begin{center}
\righthyphenmin2 \sloppy
\begin{tabular}{|p{0.33\columnwidth}|p{0.33\columnwidth}|p{0.33\columnwidth}|}
\hline
\bf Track A & \bf Track B & \bf Track C \\\hline
\it Semantics (Long +TACL Papers) & \it Information Extraction (Long Papers) & \it Computational Psycholinguistics / Machine Translation (Long Papers) \\
\TrackALoc & \TrackBLoc & \TrackCLoc \\
\hline\hline
  \marginnote{\rotatebox{90}{10:30}}[2mm]
{}\papertableentry{papers-262} & {}\papertableentry{papers-165} & {}\papertableentry{papers-488}
  \\
  \hline
  \marginnote{\rotatebox{90}{10:55}}[2mm]
{}\papertableentry{papers-491} & {}\papertableentry{papers-195} & {}\papertableentry{papers-539}
  \\
  \hline
  \marginnote{\rotatebox{90}{11:20}}[2mm]
{}\papertableentry{papers-667} & {}\papertableentry{papers-553} & {}\papertableentry{papers-121}
  \\
  \hline
  \marginnote{\rotatebox{90}{11:45}}[2mm]
{[TACL] }\papertableentry{tacl-final-009} & {}\papertableentry{papers-579} & {}\papertableentry{papers-565}
  \\
\hline\end{tabular}\end{center}

\bigskip{}
\noindent \textbf{Track D:} \emph{Long +TACL Paper Posters} \hfill \emph{\sessionchair{David}{McClosky}}\smallskip{}

\noindent {\PosterLoc \hfill \emph{10:30--12:10}}
\noindent \rule[0.5ex]{1\columnwidth}{1pt}
\begin{itemize}
\item []\textbf{Poster Cluster 1: Word Segmentation, Tagging and Parsing (P1-6)}
\item \posterlistentry{papers-316}{}
\item \posterlistentry{papers-247}{}
\item \posterlistentry{papers-406}{}
\item \posterlistentry{papers-442}{}
\item \posterlistentry{papers-472}{}
\item \posterlistentry{tacl-final-015}{[TACL] }
\end{itemize}

\bigskip{}
\noindent \textbf{Track E:} \emph{Short Paper Posters} \hfill \emph{\sessionchair{Jun-Ping}{Ng}}\smallskip{}

\noindent {\PosterLoc \hfill \emph{10:30--12:10}}
\noindent \rule[0.5ex]{1\columnwidth}{1pt}
\begin{itemize}
\item []\textbf{Poster Cluster 1: Spoken Language Processing (P1-3)}
\item \posterlistentry{papers-908}{}
\item \posterlistentry{papers-1032}{}
\item \posterlistentry{papers-1211}{}
\item []\textbf{Poster Cluster 2: Summarization (P4-18)}
\item \posterlistentry{papers-806}{}
\item \posterlistentry{papers-812}{}
\item \posterlistentry{papers-941}{}
\item \posterlistentry{papers-952}{}
\item \posterlistentry{papers-1051}{}
\item \posterlistentry{papers-1060}{}
\item \posterlistentry{papers-1066}{}
\item \posterlistentry{papers-1081}{}
\item \posterlistentry{papers-1104}{}
\item \posterlistentry{papers-1133}{}
\item \posterlistentry{papers-1134}{}
\item \posterlistentry{papers-1161}{}
\item \posterlistentry{papers-1189}{}
\item \posterlistentry{papers-1203}{}
\item \posterlistentry{papers-1251}{}
\end{itemize}

\clearpage

\newpage
\section*{Abstracts: Session 7}
\bigskip{}
\noindent{\bfseries\large Session 7A: Semantics (Long +TACL Papers)}\par
\noindent\TrackALoc\hfill\sessionchair{Luke}{Zettlemoyer}\par
\bigskip{}
\paperabstract{papers-262}{10:30--10:55}{}
\paperabstract{papers-491}{10:55--11:20}{}
\paperabstract{papers-667}{11:20--11:45}{}
\paperabstract{tacl-final-009}{11:45--12:10}{[TACL] }
\clearpage
\noindent{\bfseries\large Session 7B: Information Extraction (Long Papers)}\par
\noindent\TrackBLoc\hfill\sessionchair{Shiqi}{Zhao}\par
\bigskip{}
\paperabstract{papers-165}{10:30--10:55}{}
\paperabstract{papers-195}{10:55--11:20}{}
\paperabstract{papers-553}{11:20--11:45}{}
\paperabstract{papers-579}{11:45--12:10}{}
\clearpage
\noindent{\bfseries\large Session 7C: Computational Psycholinguistics / Machine Translation (Long Papers)}\par
\noindent\TrackCLoc\hfill\sessionchair{Nizar}{Habash}\par
\bigskip{}
\paperabstract{papers-488}{10:30--10:55}{}
\paperabstract{papers-539}{10:55--11:20}{}
\paperabstract{papers-121}{11:20--11:45}{}
\paperabstract{papers-565}{11:45--12:10}{}
\clearpage


\noindent{\bfseries\large Session 7D: Long +TACL Paper Posters} \hfill \emph{\sessionchair{David}{McClosky}}\par
\noindent{\PosterLoc \hfill \emph{10:30--12:10}}\par
\bigskip{}
\posterabstract{papers-316}{}
\posterabstract{papers-247}{}
\posterabstract{papers-406}{}
\posterabstract{papers-442}{}
\posterabstract{papers-472}{}
\posterabstract{tacl-final-015}{[TACL] }
\clearpage
\noindent{\bfseries\large Session 7E: Short Paper Posters} \hfill \emph{\sessionchair{Jun-Ping}{Ng}}\par
\noindent{\PosterLoc \hfill \emph{10:30--12:10}}\par
\bigskip{}
\posterabstract{papers-908}{}
\posterabstract{papers-1032}{}
\posterabstract{papers-1211}{}
\posterabstract{papers-806}{}
\posterabstract{papers-812}{}
\posterabstract{papers-941}{}
\posterabstract{papers-952}{}
\posterabstract{papers-1051}{}
\posterabstract{papers-1060}{}
\posterabstract{papers-1066}{}
\posterabstract{papers-1081}{}
\posterabstract{papers-1104}{}
\posterabstract{papers-1133}{}
\posterabstract{papers-1134}{}
\posterabstract{papers-1161}{}
\posterabstract{papers-1189}{}
\posterabstract{papers-1203}{}
\posterabstract{papers-1251}{}
\clearpage

\section[Session 8]{Session 8 Overview}
\begin{center}
\righthyphenmin2 \sloppy
\begin{tabular}{|p{0.33\columnwidth}|p{0.33\columnwidth}|p{0.33\columnwidth}|}
\hline
\bf Track A & \bf Track B & \bf Track C \\\hline
\it Fun and Quirky Topics (Short Papers) & \it Semantics (Short Papers) & \it Statistical Models and Machine Learning Methods / Machine Translation (Short Papers) \\
\TrackALoc & \TrackBLoc & \TrackCLoc \\
\hline\hline
  \marginnote{\rotatebox{90}{13:30}}[2mm]
{}\papertableentry{papers-838} & {}\papertableentry{papers-835} & {}\papertableentry{papers-1311}
  \\
  \hline
  \marginnote{\rotatebox{90}{13:45}}[2mm]
{}\papertableentry{papers-760} & {}\papertableentry{papers-789} & {}\papertableentry{papers-755}
  \\
  \hline
  \marginnote{\rotatebox{90}{14:00}}[2mm]
{}\papertableentry{papers-853} & {}\papertableentry{papers-858} & {}\papertableentry{papers-1233}
  \\
  \hline
  \marginnote{\rotatebox{90}{14:15}}[2mm]
{}\papertableentry{papers-1302} & {}\papertableentry{papers-1145} & {}\papertableentry{papers-1295}
  \\
  \hline
  \marginnote{\rotatebox{90}{14:30}}[2mm]
{}\papertableentry{papers-873} & {}\papertableentry{papers-1231} & {}\papertableentry{papers-744}
  \\
  \hline
  \marginnote{\rotatebox{90}{14:45}}[2mm]
{}\papertableentry{papers-1352} & {}\papertableentry{papers-1422} & {}\papertableentry{papers-960}
  \\
  \hline
  \marginnote{\rotatebox{90}{15:00}}[2mm]
{}\papertableentry{papers-1109} &  & {}\papertableentry{papers-1294}
  \\
\hline\end{tabular}\end{center}

\bigskip{}
\noindent \textbf{Track D:} \emph{Long Paper Posters} \hfill \emph{\sessionchair{Tim}{Baldwin}}\smallskip{}

\noindent {\PosterLoc \hfill \emph{13:30--15:15}}
\noindent \rule[0.5ex]{1\columnwidth}{1pt}
\begin{itemize}
\item []\textbf{Poster Cluster 1: NLP for Web and Social Media, including Computational Social Science (P1-6)}
\item \posterlistentry{papers-228}{}
\item \posterlistentry{papers-249}{}
\item \posterlistentry{papers-459}{}
\item \posterlistentry{papers-481}{}
\item \posterlistentry{papers-598}{}
\item \posterlistentry{papers-621}{}
\item []\textbf{Poster Cluster 2: Discourse (P7-9)}
\item \posterlistentry{papers-363}{}
\item \posterlistentry{papers-587}{}
\item \posterlistentry{papers-677}{}
\end{itemize}

\bigskip{}
\noindent \textbf{Track E:} \emph{Short Paper Posters} \hfill \emph{\sessionchair{Owen}{Rambow}}\smallskip{}

\noindent {\PosterLoc \hfill \emph{13:30--15:15}}
\noindent \rule[0.5ex]{1\columnwidth}{1pt}
\begin{itemize}
\item []\textbf{Poster Cluster 1: Discourse (P1-9)}
\item \posterlistentry{papers-276}{}
\item \posterlistentry{papers-748}{}
\item \posterlistentry{papers-799}{}
\item \posterlistentry{papers-880}{}
\item \posterlistentry{papers-1041}{}
\item \posterlistentry{papers-1044}{}
\item \posterlistentry{papers-1056}{}
\item \posterlistentry{papers-1091}{}
\item \posterlistentry{papers-1115}{}
\item []\textbf{Poster Cluster 2: Phonology, Morphology and Word Segmentation (P10-15)}
\item \posterlistentry{papers-815}{}
\item \posterlistentry{papers-816}{}
\item \posterlistentry{papers-894}{}
\item \posterlistentry{papers-988}{}
\item \posterlistentry{papers-1391}{}
\item \posterlistentry{papers-1460}{}
\end{itemize}

\clearpage

\newpage
\section*{Abstracts: Session 8}
\bigskip{}
\noindent{\bfseries\large Session 8A: Fun and Quirky Topics (Short Papers)}\par
\noindent\TrackALoc\hfill\sessionchair{Yanjun}{Ma}\par
\bigskip{}
\paperabstract{papers-838}{13:30--13:45}{}
\paperabstract{papers-760}{13:45--14:00}{}
\paperabstract{papers-853}{14:00--14:15}{}
\paperabstract{papers-1302}{14:15--14:30}{}
\paperabstract{papers-873}{14:30--14:45}{}
\paperabstract{papers-1352}{14:45--15:00}{}
\paperabstract{papers-1109}{15:00--15:15}{}
\clearpage
\noindent{\bfseries\large Session 8B: Semantics (Short Papers)}\par
\noindent\TrackBLoc\hfill\sessionchair{Marco}{Baroni}\par
\bigskip{}
\paperabstract{papers-835}{13:30--13:45}{}
\paperabstract{papers-789}{13:45--14:00}{}
\paperabstract{papers-858}{14:00--14:15}{}
\paperabstract{papers-1145}{14:15--14:30}{}
\paperabstract{papers-1231}{14:30--14:45}{}
\paperabstract{papers-1422}{15:00--15:15}{}
\clearpage
\noindent{\bfseries\large Session 8C: Statistical Models and Machine Learning Methods / Machine Translation (Short Papers)}\par
\noindent\TrackCLoc\hfill\sessionchair{Kevin}{Gimpel}\par
\bigskip{}
\paperabstract{papers-1311}{13:30--13:45}{}
\paperabstract{papers-755}{13:45--14:00}{}
\paperabstract{papers-1233}{14:00--14:15}{}
\paperabstract{papers-1295}{14:15--14:30}{}
\paperabstract{papers-744}{14:30--14:45}{}
\paperabstract{papers-960}{14:45--15:00}{}
\paperabstract{papers-1294}{15:00--15:15}{}
\clearpage


\noindent{\bfseries\large Session 8D: Long Paper Posters} \hfill \emph{\sessionchair{Tim}{Baldwin}}\par
\noindent{\PosterLoc \hfill \emph{13:30--15:15}}\par
\bigskip{}
\posterabstract{papers-228}{}
\posterabstract{papers-249}{}
\posterabstract{papers-459}{}
\posterabstract{papers-481}{}
\posterabstract{papers-598}{}
\posterabstract{papers-621}{}
\posterabstract{papers-363}{}
\posterabstract{papers-587}{}
\posterabstract{papers-677}{}
\clearpage
\noindent{\bfseries\large Session 8E: Short Paper Posters} \hfill \emph{\sessionchair{Owen}{Rambow}}\par
\noindent{\PosterLoc \hfill \emph{13:30--15:15}}\par
\bigskip{}
\posterabstract{papers-276}{}
\posterabstract{papers-748}{}
\posterabstract{papers-799}{}
\posterabstract{papers-880}{}
\posterabstract{papers-1041}{}
\posterabstract{papers-1044}{}
\posterabstract{papers-1056}{}
\posterabstract{papers-1091}{}
\posterabstract{papers-1115}{}
\posterabstract{papers-815}{}
\posterabstract{papers-816}{}
\posterabstract{papers-894}{}
\posterabstract{papers-988}{}
\posterabstract{papers-1391}{}
\posterabstract{papers-1460}{}
\clearpage

\section[Session 9]{Session 9 Overview}
\begin{center}
\righthyphenmin2 \sloppy
\begin{tabular}{|p{0.33\columnwidth}|p{0.33\columnwidth}|p{0.33\columnwidth}|}
\hline
\bf Track A & \bf Track B & \bf Track C \\\hline
\it Statistical Models and Machine Learning Methods (Long + TACL Papers) & \it Text Mining and NLP Applications (Long Papers) & \it Spoken Language Processing and Language Modeling (Long Papers) \\
\TrackALoc & \TrackBLoc & \TrackCLoc \\
\hline\hline
  \marginnote{\rotatebox{90}{15:40}}[2mm]
{}\papertableentry{papers-060} & {}\papertableentry{papers-147} & {}\papertableentry{papers-258}
  \\
  \hline
  \marginnote{\rotatebox{90}{16:05}}[2mm]
{}\papertableentry{papers-323} & {}\papertableentry{papers-454} & {}\papertableentry{papers-580}
  \\
  \hline
  \marginnote{\rotatebox{90}{16:30}}[2mm]
{}\papertableentry{papers-588} & {}\papertableentry{papers-660} & {}\papertableentry{papers-638}
  \\
  \hline
  \marginnote{\rotatebox{90}{16:55}}[2mm]
{[TACL] }\papertableentry{tacl-final-011} & {}\papertableentry{papers-315} & {}\papertableentry{papers-382}
  \\
\hline\end{tabular}\end{center}

\bigskip{}
\noindent \textbf{Track D:} \emph{Long Paper Posters} \hfill \emph{\sessionchair{Nathan}{Schneider}}\smallskip{}

\noindent {\PosterLoc \hfill \emph{15:40--17:20}}
\noindent \rule[0.5ex]{1\columnwidth}{1pt}
\begin{itemize}
\item []\textbf{Poster Cluster 1: Semantics (P1-8)}
\item \posterlistentry{papers-105}{}
\item \posterlistentry{papers-109}{}
\item \posterlistentry{papers-116}{}
\item \posterlistentry{papers-181}{}
\item \posterlistentry{papers-185}{}
\item \posterlistentry{papers-204}{}
\item \posterlistentry{papers-243}{}
\item \posterlistentry{papers-308}{}
\end{itemize}

\bigskip{}
\noindent \textbf{Track E:} \emph{Short Paper Posters} \hfill \emph{\sessionchair{Kang}{Liu}}\smallskip{}

\noindent {\PosterLoc \hfill \emph{15:40--17:20}}
\noindent \rule[0.5ex]{1\columnwidth}{1pt}
\begin{itemize}
\item []\textbf{Poster Cluster 1: Sentiment Analysis and Opinion Mining (P1-9)}
\item \posterlistentry{papers-772}{}
\item \posterlistentry{papers-796}{}
\item \posterlistentry{papers-832}{}
\item \posterlistentry{papers-1040}{}
\item \posterlistentry{papers-1065}{}
\item \posterlistentry{papers-1090}{}
\item \posterlistentry{papers-1313}{}
\item \posterlistentry{papers-1331}{}
\item \posterlistentry{papers-1387}{}
\item []\textbf{Poster Cluster 2: NLP for the Web and Social Media, including Computational Social Science (P10-17)}
\item \posterlistentry{papers-943}{}
\item \posterlistentry{papers-983}{}
\item \posterlistentry{papers-994}{}
\item \posterlistentry{papers-1139}{}
\item \posterlistentry{papers-1149}{}
\item \posterlistentry{papers-1188}{}
\item \posterlistentry{papers-1361}{}
\end{itemize}

\clearpage

\newpage
\section*{Abstracts: Parallel Session 9}
\bigskip{}
\noindent{\bfseries\large Session 9A: Statistical Models and Machine Learning Methods (Long + TACL Papers)}\par
\noindent\TrackALoc\hfill\sessionchair{Jason}{Eisner}\par
\bigskip{}
\paperabstract{papers-060}{15:40--16:05}{}
\paperabstract{papers-323}{16:05--16:30}{}
\paperabstract{papers-588}{16:30--16:55}{}
\paperabstract{tacl-final-011}{16:55--17:20}{[TACL] }
\clearpage
\noindent{\bfseries\large Session 9B: Text Mining and NLP Applications (Long Papers)}\par
\noindent\TrackBLoc\hfill\sessionchair{Shuming}{Shi}\par
\bigskip{}
\paperabstract{papers-147}{15:40--16:05}{}
\paperabstract{papers-454}{16:05--16:30}{}
\paperabstract{papers-660}{16:30--16:55}{}
\paperabstract{papers-315}{16:55--17:20}{}
\clearpage
\noindent{\bfseries\large Session 9C: Spoken Language Processing and Language Modeling (Long Papers)}\par
\noindent\TrackCLoc\hfill\sessionchair{Isabel}{Trancoso}\par
\bigskip{}
\paperabstract{papers-258}{15:40--16:05}{}
\paperabstract{papers-580}{16:05--16:30}{}
\paperabstract{papers-638}{16:30--16:55}{}
\paperabstract{papers-382}{16:55--17:20}{}
\clearpage


\noindent{\bfseries\large Session 9D: Long Paper Posters} \hfill \emph{\sessionchair{Nathan}{Schneider}}\par
\noindent{\PosterLoc \hfill \emph{15:40--17:20}}\par
\bigskip{}
\posterabstract{papers-105}{}
\posterabstract{papers-109}{}
\posterabstract{papers-116}{}
\posterabstract{papers-181}{}
\posterabstract{papers-185}{}
\posterabstract{papers-204}{}
\posterabstract{papers-243}{}
\posterabstract{papers-308}{}
\clearpage
\noindent{\bfseries\large Session 9E: Short Paper Posters} \hfill \emph{\sessionchair{Kang}{Liu}}\par
\noindent{\PosterLoc \hfill \emph{15:40--17:20}}\par
\bigskip{}
\posterabstract{papers-772}{}
\posterabstract{papers-796}{}
\posterabstract{papers-832}{}
\posterabstract{papers-1040}{}
\posterabstract{papers-1065}{}
\posterabstract{papers-1090}{}
\posterabstract{papers-1313}{}
\posterabstract{papers-1331}{}
\posterabstract{papers-1387}{}
\posterabstract{papers-943}{}
\posterabstract{papers-983}{}
\posterabstract{papers-994}{}
\posterabstract{papers-1139}{}
\posterabstract{papers-1149}{}
\posterabstract{papers-1188}{}
\posterabstract{papers-1361}{}
\clearpage

