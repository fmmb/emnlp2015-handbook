Much of what we understand from text is not explicitly stated. Rather, the reader uses his/her knowledge to fill in gaps and create a coherent, mental picture or ``scene'' depicting what text appears to convey. The scene constitutes an understanding of the text, and can be used to answer questions that go beyond the text. Our goal is to answer elementary science questions, where this requirement is pervasive; A question will often give a partial description of a scene and ask the student about implicit information. We show that by using a simple ``knowledge graph'' representation of the question, we can leverage several large-scale linguistic resources to provide missing background knowledge, somewhat alleviating the knowledge bottleneck in previous approaches. The coherence of the best resulting scene, built from a question/answer-candidate pair, reflects the confidence that the answer candidate is correct, and thus can be used to answer multiple choice questions. Our experiments show that this approach significantly outperforms competitive algorithms on several datasets tested. The significance of this work is thus to show that a simple ``knowledge graph'' representation allows a version of ``interpretation as scene construction'' to be made viable.
