Semantic data regarding points of interest in urban areas are hard to visualize. Due to the high number of points and categories they belong, as well as the associated textual information, maps become heavily cluttered and hard to read. Using traditional visualization techniques (e.g. dot distribution maps, typographic maps) partially solve this problem. Although, these techniques address different issues of the problem, their combination is hard and typically results in an efficient visualization. In our approach, we present a method to represent clusters of points of interest as shapes, which is based on vacuum package metaphor. The calculated shapes characterize sets of points and allow their use as containers for textual information. Additionally, we present a strategy for placing text onto polygons. The suggested method can be used in interactive visual exploration of semantic data distributed in space, and for creating maps with similar characteristics of dot distribution maps, but using shapes instead of points.
