In November 2014, the European Central Bank (ECB) started to directly supervise the largest banks in the Eurozone via the Single Supervisory Mechanism (SSM). While supervisory risk assessments are usually based on quantitative data and surveys, this work explores whether sentiment analysis is capable of measuring a bank's attitude and opinions towards risk by analyzing text data. For realizing this study, a collection consisting of more than 500 CEO letters and outlook sections extracted from bank annual reports is built up. Based on these data, two distinct experiments are conducted. The evaluations find promising opportunities, but also limitations for risk sentiment analysis in banking supervision. At the level of individual banks, predictions are relatively inaccurate. In contrast, the analysis of aggregated figures revealed strong and significant correlations between uncertainty or negativity in textual disclosures and the quantitative risk indicator's future evolution. Risk sentiment analysis should therefore rather be used for macroprudential analyses than for assessments of individual banks.
