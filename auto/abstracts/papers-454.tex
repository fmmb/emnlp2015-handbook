In a typical social media content analysis task, the user is interested in analyzing posts of a particular topic. Identifying such posts is often formulated as a classification problem. However, this problem is challenging. One key issue is covariate shift. That is, the training data is not fully representative of the test data. We observed that the covariate shift mainly occurs in the negative data because topics discussed in social media are highly diverse and numerous, but the user-labeled negative training data may cover only a small number of topics. This paper proposes a novel technique to solve the problem. The key novelty of the technique is the transformation of document representation from the traditional n-gram feature space to a center-based similarity (CBS) space. In the CBS space, the covariate shift problem is significantly mitigated, which enables us to build much better classifiers. Experiment results show that the proposed approach markedly improves classification.
