Discourse relations are a bridge between sentence-level semantics and discourse-level semantics. They can be signalled explicitly with discourse connectives or conveyed implicitly, to be inferred by a comprehender. The same discourse units can be related in more than one way, signalled by multiple connectives. But multiple connectives aren't necessary: Multiple relations can be conveyed even when only one connective is explicit. This paper describes the initial phase in a larger experimental study aimed at answering two questions: (1) Given an explicit discourse adverbial, what discourse relation(s) do naive subjects take to be operative, and (2) Can this be predicted on the basis of the explicit adverbial alone, or does it depend instead on other factors?
