The rise in popularity and ubiquity of Twitter has made sentiment analysis of tweets an important and well-covered area of research. However, the 140 character limit imposed on tweets makes it hard to use standard linguistic methods for sentiment classification. On the other hand, what tweets lack in structure they make up with sheer volume and rich metadata. This metadata includes geolocation, temporal and author information. We hypothesize that sentiment is dependent on all these contextual factors. Different locations, times and authors have different emotional valences. In this paper, we explored this hypothesis by utilizing distant supervision to collect millions of labelled tweets from different locations, times and authors. We used this data to analyse the variation of tweet sentiments across different authors, times and locations. Once we explored and understood the relationship between these variables and sentiment, we used a Bayesian approach to combine these variables with more standard linguistic features such as n-grams to create a Twitter sentiment classifier. This combined classifier outperforms the purely linguistic classifier, showing that integrating the rich contextual information available on Twitter into sentiment classification is a promising direction of research.
