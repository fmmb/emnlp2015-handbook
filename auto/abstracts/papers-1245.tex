Wikipedia is the largest collection of encyclopedic data ever written in the history of humanity. Thanks to its coverage and its availability in machine-readable format, it has become a primary resource for large-scale research in historical and cultural studies. In this work, we focus on the subset of pages describing persons, and we investigate the task of recognizing biographical sections from them: given a person's page, we identify the list of sections where information about her/his life is present. We model this as a sequence classification problem, and propose a supervised setting, in which the training data are acquired automatically. Besides, we show that six simple features extracted only from the section titles are very informative and yield good results well above a strong baseline.
