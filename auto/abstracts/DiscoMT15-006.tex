Previous work on pronouns in SMT has focussed on third-person pronouns, treating them all as anaphoric. Little attention has been paid to other uses or other types of pronouns. Believing that further progress requires careful analysis of pronouns as a whole, we have analysed a parallel corpus of annotated English-German texts to highlight some of the problems that hinder progress. We combine this with an assessment of the ability of two state-of-the-art systems to translate different pronoun types.
