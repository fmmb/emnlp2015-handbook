Distantly supervised approaches have become popular in recent years as they allow training relation extractors without text-bound annotation, using instead known relations from a knowledge base and a large textual corpus from an appropriate domain. While state of the art distant supervision approaches use off-the-shelf named entity recognition and classification NERC systems to identify relation arguments,  discrepancies in domain or genre between the data used for NERC training and the intended domain for the relation extractor can lead to low performance. This is particularly problematic for ``non-standard'' named entities such as album which would fall into the MISC category. We propose to ameliorate this issue by jointly training the named entity classifier and the relation extractor using imitation learning which reduces structured prediction learning to classification learning. We further experiment with Web features different features and compare against using two off-the-shelf supervised NERC systems, Stanford NER and FIGER, for named entity classification. Our experiments show that imitation learning improves average precision by 4 points over an one-stage classification model, while removing Web features results in a 6 points reduction. Compared to using FIGER and Stanford NER, average precision is 10 points and 19 points higher with our imitation learning approach.
