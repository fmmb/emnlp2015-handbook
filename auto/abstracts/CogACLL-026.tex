Redundancy is an important psycholinguistic concept which is often used for explanations of language change, but is notoriously diffi-cult to operationalize and measure. Assum-ing that the reconstruction of a syntactic structure by a parser can be used as a rough model of the understanding of a sentence by a human hearer, I propose a method for es-timating redundancy. The key idea is to compare performances of a parser on a giv-en treebank before and after artificially re-moving all information about a certain grammeme from the morphological annota-tion. The change in performance can be used as an estimate for the redundancy of the grammeme. I perform an experiment, apply-ing MaltParser to an Old Church Slavonic treebank to estimate grammeme redundancy in Proto-Slavic. The results show that those Old Church Slavonic grammemes within the case, number and tense categories that were estimated as most redundant are those that disappeared in modern Russian. Moreover, redundancy estimates serve as a good predic-tor of case grammeme frequencies in modern Russian. The small size of the samples do not allow to make definitive conclusions for number and tense.
