In this paper, a turn-taking phenomenon taxonomy is introduced, organised accord- ing to the level of information conveyed. It is aimed to provide a better grasp of the behaviours used by humans while talk- ing to each other, so that they can be methodically replicated in dialogue sys- tems. Five interesting phenomena have been implemented in a simulated environ- ment: the system barge-in because of an unclear, an incoherent or a sufficient mes- sage, the feedback and the user barge-in. The aim of the experiment is to illustrate that some phenomena are worth imple- menting in some cases and others are not.
