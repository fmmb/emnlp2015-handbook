Community question answering, a recent evolution of question answering in the Web context, allows a user to quickly consult the opinion of a number of people on a particular topic, thus taking advantage of the wisdom of the crowd. Here we try to help the user by deciding automatically which answers are good and which are bad for a given question. In particular, we focus on exploiting the output structure at the thread level in order to make more consistent global decisions. More specifically, we exploit the relations between pairs of comments at any distance in the thread, which we incorporate in a graph-cut and in an ILP frameworks. The evaluation on the benchmark dataset of SemEval-2015 Task 3 confirms the importance of using thread-level information, which allows us to improve over the state of the art.
