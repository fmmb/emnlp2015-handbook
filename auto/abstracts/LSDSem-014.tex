The main contribution of this paper is a cross-linguistic empirical analysis of two interacting levels of linguistic analysis of written text: situation entities (SEs), the semantic types of situations evoked by clauses of text, and discourse modes (DMs), a characterization of passages at the sub-document level. We adapt an existing annotation scheme for SEs in English to be used for German data, with a detailed discussion of the most important differences. We create the first parallel corpus annotated for SEs, and the first DM-annotated corpus. We find that: (a) the adapted scheme is supported by evidence from a large-scale experimental study; (b) SEs mainly correspond to each other in parallel text, and a large part of the mismatches are systematic; (c) the DM annotation task can be performed intuitively with reasonable agreement; and (d) the annotated DMs show the predicted differences in the distributions of SE types.
