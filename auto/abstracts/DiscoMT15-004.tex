In this paper, we analyse cross-linguistic variation of discourse phenomena, i.e. coreference, discourse relations and modality. We will show that contrasts in the distribution of these phenomena can be observed across languages, genres, and text production types, i.e. translated and non-translated ones. Translations, regardless of the method they were produced with, are different from their source texts and from the comparable originals in the target language, as it was stated in studies on translationese. These differences can be automatically detected and analysed with exploratory and automatic clustering techniques. The extracted frequency-based profiles of variables under analysis (languages, genres, text production types) can be used in further studies, e.g. in the development and enhancement of MT systems, or in further NLP applications.
