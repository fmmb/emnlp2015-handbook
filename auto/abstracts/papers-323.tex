This paper proposes a tree-based convolutional neural network (TBCNN) for discriminative sentence modeling. Our model leverages either constituency trees or dependency trees. The tree-based convolution process extracts sentences' structural features, which are then aggregated by max pooling. Such architecture allows short propagation paths between the output layer and underlying feature detectors, enabling effective structural feature learning and extraction. We evaluate our models on two tasks: sentiment analysis and question classification. In both experiments, TBCNN outperforms previous state-of-the-art results, including existing neural networks and dedicated feature/rule engineering. We also make efforts to visualize the tree-based convolution process, shedding some light on how our models work.
