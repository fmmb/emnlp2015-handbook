Humor is an essential component in personal communication. How to create computational models to discover the structure behind humor, recognize humor and even extract humor anchors remains a challenge. In this work, we first identify several semantic structures behind humor and design sets of features for each theory, and next employ a computational approach to recognize humor. Furthermore, we develop a simple and effective method to extract anchors that enable humor in a sentence. Experiments conducted on two datasets demonstrate that our humor recognizer is effective in automatically distinguishing between humorous and non-humorous texts and our extracted humor anchors correlate quite well with human annotations.
