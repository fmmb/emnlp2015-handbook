Usage of discourse connectives (DCs) differs across languages, thus addition and omission of connectives are common in translation. We investigate how implicit (omitted) DCs in the source text impacts various machine translation (MT) systems, and whether a discourse parser is needed as a preprocessor to explicitate implicit DCs. Based on the manual annotation and alignment of 7266 pairs of discourse relations in a Chinese-English translation corpus, we evaluate whether a preprocessing step that inserts explicit DCs at positions of implicit relations can improve MT. Results show that, without modifying the translation model, explicitating implicit relations in the input source text has limited effect on MT evaluation scores. In addition, translation spotting analysis shows that it is crucial to identify DCs that should be explicitly translated in order to improve implicit-to-explicit DC translation. On the other hand, further analysis reveals that the disambiguation as well as explicitation of implicit relations are subject to a certain level of optionality, suggesting the limitation to learn and evaluate this linguistic phenomenon using standard parallel corpora.
