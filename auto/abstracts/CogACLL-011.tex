We address the question whether children can acquire mature use of higher-level grammatical choices from the linguistic input, given only general prior knowledge and learning biases. We do so on the basis of a case study with the dative alternation in English, building on a study by De Marneffe et al. (2012) who model the production of the dative alternation by seven young children, using data from the Child Language Data Exchange System corpus. Using mixed-effects logistic modelling on the aggregated data of these  children, De Marneffe et al. report that the children's choices can be predicted both by their own utterances and by child-directed speech. Here we bring the computational modeling down to the individual child, using memory-based learning and incremental learning curve studies. We observe that for all children, their dative choices are best predicted by a model trained on child-directed speech. Yet, models trained on two individual children for which sufficient data is available are about as accurate. Furthermore, models trained on the dative alternations of these children provide approximations of dative alternations in caregiver speech that are about as accurate as training and testing on caregiver data only.
