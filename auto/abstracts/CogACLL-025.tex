Meaning conveyance is bottlenecked by the linguistic conventions shared among interlocutors. One possibility to convey non-conventionalized meaning is to employ known expressions in such a way that the intended meaning can be abduced from them. This, in turn, can give rise to ambiguity. We investigate this process with a focus on its use for semantic coordination and show it to be conducive to fast agreement on novel meaning under a mutual expectation to exploit semantic structure. We argue this to be a motivation for the cross-linguistic pervasiveness of systematic ambiguity.
