The use of Electronic Health Records (EHRs) is becoming more prevalent in healthcare institutions world-wide. These digital records contain a wealth of information on patients' health in the form of Natural Language text. The electronic format of the clinical notes has evident advantages in terms of storage and shareability, but also makes it easy to duplicate information from one document to another through copy-pasting. Previous studies have shown that (copy-paste-induced) redundancy can reach high levels in American EHRs, and that these high levels of redundancy have a negative effect on the performance of Natural Language Processing (NLP) tools that are used to process EHRs automatically. In this paper, we present a preliminary study on the level of redundancy in French EHRs. We study the evolution of redundancy over time, and its occurrence in respect to different document types and sections in a small corpus comprising of three patient records (361 documents). We find that average redundancy levels in our subset are lower than those observed in U.S. corpora (respectively 33\% vs. up to 78\%), which may indicate different cultural practices between these two countries. Moreover, we find no evidence of the incremental increase (over time) of redundant text in clinical notes which has been found in American EHRs. These results suggest that redundancy mitigating strategies may not be needed when processing French EHRs.
