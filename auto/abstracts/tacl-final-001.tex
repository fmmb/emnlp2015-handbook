Annotated data is prerequisite for many NLP applications. Acquiring large-scale annotated corpora is a major bottleneck, requiring significant time and resources. Recent work has proposed turning annotation into a game to increase its appeal and lower its cost; however, current games are largely text-based and closely resemble traditional an notation tasks. We propose a new linguistic annotation paradigm that produces annotations from playing graphical video games. The effectiveness of this design is demonstrated using two video games: one to create a mapping from WordNet senses to images, and a second game that performs Word Sense Disambiguation. Both games produce accurate results. The first game yields annotation quality equal to that of experts and a cost reduction of 73\% over equivalent crowdsourcing; the second game provides a 16.3\% improvement in accuracy over current state-of-the-art sense disambiguation games with WordNet.
