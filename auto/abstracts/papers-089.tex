Cross-Lingual Learning provides a mechanism to adapt NLP tools available for label rich languages to achieve similar tasks for label-scarce languages. An efficient cross-lingual tool significantly reduces the cost and effort required to manually annotate data. In this paper, we use the Recursive Autoencoder architecture to develop a Cross Lingual Sentiment Analysis (CLSA) tool using sentence aligned corpora between a pair of resource rich (English) and resource poor(Hindi) language. The system is based on the assumption that semantic similarity between different phrases also implies sentiment similarity in majority of sentences. The resulting system is then analyzed on a newly developed Movie Reviews Dataset in Hindi with labels given on a rating scale and compare performance of our system against existing systems. It is shown that our approach significantly outperforms state of the art systems for Sentiment Analysis, especially when labeled data is scarce.
