In this paper, we introduce the notion of visually descriptive language (VDL) -- intuitively a text segment whose truth can be confirmed by visual sense alone. VDL can be exploited in many vision-based tasks, e.g. image interpretation and story illustration. In contrast to previous work requiring pre-aligned texts and images, we propose a broader definition of VDL that extends to a much larger range of texts without associated images. We also discuss possible VDL annotation tasks and make recommendations for difficult cases. Lastly, we demonstrate the viability of our definition via an annotation exercise across several text genres and analyse inter-annotator agreement. Results show reasonably high levels of agreement between annotators can be reached.
