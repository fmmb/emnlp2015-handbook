Twitter is often used in quantitative studies that identify geographically-preferred topics, writing styles, and entities. These studies rely on either GPS coordinates attached to individual messages, or on the user-supplied location field in each profile. In this paper, we compare these data acquisition techniques and quantify the biases that they introduce; we also measure their effects on linguistic analysis and text-based geolocation. GPS-tagging and self-reported locations yield measurably different corpora, and these linguistic differences are partially attributable to differences in dataset composition by age and gender. Using a latent variable model to induce age and gender, we show how these demographic variables interact with geography to affect language use. We also show that the accuracy of text-based geolocation varies with population demographics, giving the best results for men above the age of 40.
