According to the principle of compositionality, the meaning of a sentence is computed from the meaning of its parts and the way they are syntactically combined. In practice, however, the syntactic structure is computed by automatic parsers which are far-from-perfect and not tuned to the specifics of the task. Current recursive neural network (RNN) approaches for computing sentence meaning therefore run into a number of practical difficulties, including the need to carefully select a parser appropriate for the task, deciding how and to what extent syntactic context modifies the semantic composition function, as well as on how to transform parse trees to conform to the branching settings (typically, binary branching) of the RNN. This paper introduces a new model, the Forest Convolutional Network, that avoids all of these challenges, by taking a parse forest as input, rather than a single tree, and by allowing arbitrary branching factors. We report improvements over the state-of-the-art in sentiment analysis and question classification.
