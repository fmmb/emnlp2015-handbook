De-identification aims at preserving patient confidentiality while enabling the use of clinical documents for furthering medical research. Herein, we aim to evaluate patient re-identification chances on a corpus of clinical documents in French. Personal Health Identifiers are automatically marked by a de-identification system applied to the corpus, followed by reintroduction of plausible surrogates. The resulting documents are shown to individuals with varying knowledge of the documents and de-identification method. The individuals are asked to re-identify the patients. The amount of information recovered increases with familiarity with the documents and/or de-identification method. Surrogate re-introduction with localization from the same (vs. different) geographical area as the original documents is found more effective. The amount of information recovered was not sufficient to re-identify any of the patients, except when privileged access to the hospital health information system and several documents about the same patient were available.
