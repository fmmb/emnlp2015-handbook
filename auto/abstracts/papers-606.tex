Sarcasm is generally characterized as a figure of speech that involves the substitution of a literal by a figurative meaning, which is usually the opposite of the original literal meaning. We re-frame the sarcasm detection task as a type of word sense disambiguation problem, where the sense of a word is either literal or sarcastic. We call this the Literal/Sarcastic Sense Disambiguation (LSSD) task. We address two issues: 1) how to collect a set of target words that  can have either literal or sarcastic meanings depending on context; and 2) given an utterance and a target word, how to automatically detect whether the target word is used in the literal or the sarcastic sense. For the latter, we investigate several distributional semantics methods and show that a Support Vector Machines (SVM) classifier with a modified kernel using word embeddings achieves a 7-10\% F1 improvement over a strong lexical baseline.
