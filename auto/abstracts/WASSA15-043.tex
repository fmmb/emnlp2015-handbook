We present a pilot study analyzing the connotative language found in a bilingual corpus of French and English headlines. We find that (1) manual annotation of connotation at the word-level is more reliable than using segment-level judgments, (2) connotation polarity is often, but not always, preserved in reference translations produced by humans, (3) machine translated text does not preserve the connotative language identified by an English connotation lexicon. These lessons will helps us build new resources to learn better models of connotation and translation.
