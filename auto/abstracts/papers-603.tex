Social media such as Twitter have become an important method of communication, with potential opportunities for NLG to facilitate the generation of social media content. We focus on the generation of indicative tweets that contain a link to an external web page. While it is natural and tempting to view the linked web page as the source text from which the tweet is generated in an extractive summarization setting, it is unclear to what extent actual indicative tweets behave like extractive summaries. We collect a corpus of indicative tweets with their associated articles and investigate to what extent they can be derived from the articles using extractive methods. We also consider the impact of the formality and genre of the article. Our results demonstrate the limits of viewing indicative tweet generation as extractive summarization, and point to the need for the development of a methodology for tweet generation that is sensitive to genre-specific issues.
