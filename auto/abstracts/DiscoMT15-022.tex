The translation process in statistical machine translation (SMT) is shaped by technical constraints and engineering considerations. SMT explicitly models translation as search for a target-language equivalent of the input text. This perspective on translation had wide currency in mid-20th century translation studies, but has since been superseded by approaches arguing for a more complex relation between source and target text. In this paper, we show how traditional assumptions of translational equivalence are embodied in SMT through the concepts of ``word alignment'' and ``domain'' and discuss some limitations arising from the word-level/corpus-level dichotomy inherent in these concepts.
