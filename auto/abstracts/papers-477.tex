The information conveyed by some sentences would be more easily understood by a reader if it were expressed in multiple sentences. We call such sentences content heavy: these are possibly grammatical but difficult to comprehend, cumbersome sentences. In this paper we introduce the task of detecting content-heavy sentences in cross-lingual context. Specifically we develop methods to identify sentences in Chinese for which English speakers would prefer translations consisting of more than one sentence.  We base our analysis and definitions on evidence from multiple human translations and reader preferences on flow and understandability. We show that machine translation quality when translating content heavy sentences is markedly worse than overall quality and that this type of sentence are fairly common in Chinese news. We demonstrate that sentence length and  punctuation usage in Chinese are not sufficient clues for accurately detecting heavy sentences and present a richer classification model that accurately identifies these sentences.
