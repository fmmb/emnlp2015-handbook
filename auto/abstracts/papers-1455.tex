Language use is known to be influenced by personality traits as well as by socio-demographic characteristics such as age or mother tongue. As a result, it is possible to automatically identify these traits of the author from her texts. It has recently been shown that knowledge of such dimensions can improve performance in NLP tasks such as topic and sentiment modeling. We posit that machine translation is another application that should be personalized. In order to motivate this, we explore whether translation preserves demographic and psychometric traits. We show that, largely, both translation of the source training data into the target language, and the target test data into the source language has a detrimental effect on the accuracy of predicting author traits. We argue that this supports the need for personal and personality-aware machine translation models.
