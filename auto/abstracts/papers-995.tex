Scientific theories and models in Earth science typically involve changing variables and their complex interactions, including correlations, causal relations and chains of positive/negative feedback loops. Variables tend to be complex rather than atomic entities and expressed as noun phrases containing multiple modifiers, e.g. ``oxygen depletion in the upper 500 m of the ocean'' or ``timing and magnitude of surface temperature evolution in the Southern Hemisphere in deglacial proxy records''. Text mining from Earth science literature is therefore significantly different from biomedical text mining and requires different approaches and methods. Our approach aims at automatically locating and extracting variables and their direction of variation: increasing, decreasing or just changing. Variables are initially extracted by matching tree patterns onto the syntax trees of the source texts. Next, variables are generalised in order to enhance their similarity, facilitating hierarchical search and inference. This generalisation is accomplished by progressive pruning of syntax trees using a set of tree transformation operations. Text mining results are presented as a browsable variable hierarchy which allows users to inspect all mentions of a particular variable type in the text as well as any generalisations or specialisations. The approach is demonstrated on a corpus of 10k abstracts of Nature publications in the field of Marine science. We discuss experiences with this early prototype and outline a number of possible improvements and directions for future research.
