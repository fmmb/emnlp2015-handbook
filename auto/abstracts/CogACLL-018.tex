Children's overextension errors in word usage can yield insights into the underlying representation of meaning. We simulate overextension patterns in the domain of color with two word-learning models, and look at the contribution of three possible factors: perceptual properties of the colors, typological prevalence of certain color groupings into categories (as a proxy for cognitive naturalness), and color term frequency. We find that the perceptual features provide the strongest predictors of the error pattern observed during development, and can effectively rule out color term frequency as an explanation. Typological prevalence is shown to correlate strongly with the perceptual dimensions of color, and hence provides no effect over and above the perceptual dimensions.
