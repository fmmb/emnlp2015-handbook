Expectation-maximization algorithms, such as those implemented in GIZA++ pervade the field of unsupervised word alignment. However, these algorithms have a problem of over-fitting, leading to ``garbage collector effects,'' where rare words tend to be erroneously aligned to untranslated words. This paper proposes a leave-one-out expectation-maximization algorithm for unsupervised word alignment to address this problem. The proposed method excludes information derived from the alignment of a sentence pair from the alignment models used to align it. This prevents erroneous alignments within a sentence pair from supporting themselves. Experimental results on Chinese-English and Japanese-English corpora show that the F\$\_1\$, precision and recall of alignment were consistently increased by 5.0\% -- 17.2\%, and BLEU scores of end-to-end translation were raised by 0.03 -- 1.30.  The proposed method also outperformed \$l\_0\$-normalized GIZA++ and Kneser-Ney smoothed GIZA++.
