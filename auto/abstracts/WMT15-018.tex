This paper describes the Universitat d'Alacant submissions (labelled as UAlacant) for the machine translation quality estimation (MTQE) shared task in WMT 2015, where we participated in the word-level MTQE sub-task. The method we used to produce our submissions uses external sources of bilingual information as a ``black box'' to spot sub-segment correspondences between a source segment S and the translation hypothesis T produced by a machine translation system. This is done by segmenting both S and T into overlapping sub-segments of variable length and translating them in both translation directions, using the available sources of bilingual information ``on the fly''. For our submissions, two sources of bilingual information were used: machine translation (Apertium and Google Translate) and the bilingual concordancer Reverso Context. After obtaining the sub-segment correspondences, a collection of features is extracted from them, which are then used by a binary classifer to obtain the final ``GOOD'' or ``BAD'' word-level quality labels. We prepared two submissions for this year's edition of WMT 2015: one using the features produced by our system, and one combining them with the baseline features published by the organisers of the task, which were ranked third and first for the sub-task, respectively.
