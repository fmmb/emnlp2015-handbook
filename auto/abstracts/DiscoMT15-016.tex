Translation of discourse connectives varies more in human translations than in machine translations. Building on Murray's (1997) continuity hypothesis and Sanders' (2005) causality-by-default hypothesis we investigate whether expectedness influences the degree of implicitation and explicitation of discourse relations. We manually analyze how source text connectives are translated, and where connectives in target texts come from. We establish whether relations are explicitly signaled in the other language as well, or whether they have to be reconstructed by inference. We demonstrate that the amount of implicitation and explicitation of connectives in translation is influenced by the expectedness of the relation a connective signals. In addition, we show that the types of connectives most often added in translation are also the ones most often deleted.
