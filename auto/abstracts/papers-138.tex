We explore some of the practicalities of using random walk inference methods, such as the Path Ranking Algorithm (PRA), for the task of knowledge base completion. We show that the random walk probabilities computed (at great expense) by PRA provide no discernible benefit to performance on this task, and so they can safely be dropped. This allows us to define a simpler algorithm for generating feature matrices from graphs, which we call subgraph feature extraction (SFE). In addition to being conceptually simpler than PRA, SFE is much more efficient, reducing computation by an order of magnitude, and more expressive, allowing for much richer features than just paths between two nodes in a graph. We show experimentally that this technique gives substantially better performance than PRA and its variants, improving mean average precision from .432 to .528 on a knowledge base completion task using the NELL knowledge base.
