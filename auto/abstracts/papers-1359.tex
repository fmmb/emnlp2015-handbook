This paper proposes a method for hierarchical phrase-based stream decoding. A stream decoder is able to take a continuous stream of tokens as input, and segments this stream into word sequences that are translated and output as a stream of target word sequences. Phrase-based stream decoding techniques have been shown to be effective as a means of simultaneous interpretation. In this paper we transfer the essence of this idea into the framework of hierarchical machine translation. The hierarchical decoding framework organizes the decoding process into a chart; this structure is naturally suited to the process of stream decoding, leading to an efficient stream decoding algorithm that searches a restricted subspace containing only relevant hypotheses. Furthermore, the decoder allows more explicit access to the word re-ordering process that is of critical importance in decoding while interpreting. The decoder was evaluated on TED talk data for English-Spanish and English-Chinese. Our results show that like the phrase-based stream decoder, the hierarchical is capable of approaching the performance of the underlying hierarchical phrase-based machine translation decoder, at useful levels of latency. In addition the hierarchical approach appeared to be robust to the difficulties presented by the more challenging English-Chinese task.
