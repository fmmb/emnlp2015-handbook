Electronic health records have emerged as a promising source of information for pharmacovigilance. Adverse drug events are, however, known to be heavily underreported, which makes it important to develop capabilities to detect such information automatically in clinical text. While machine learning offers possible solutions, it remains unclear how best to represent clinical notes in a manner conducive to learning high-performing predictive models. Here, 42 representations are explored in an empirical investigation using 27 real, clinical datasets, indicating that combining local and global (distributed) representations of words and named entities yields higher accuracy than using either in isolation. Subsequent analyses highlight the relative importance of various named entity classes for predicting adverse drug events.
