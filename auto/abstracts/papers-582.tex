Elementary-level science exams pose significant knowledge acquisition and reasoning challenges for automatic question answering. We develop a system that reasons with knowledge derived from textbooks, represented in a subset of first-order logic. Automatic extraction, while scalable, often results in knowledge that is incomplete and noisy, motivating use of reasoning mechanisms that handle uncertainty. Markov Logic Networks (MLNs) seem a natural model for expressing such knowledge, but the exact way of leveraging MLNs is by no means obvious. We investigate three ways of applying MLNs to our task. First, we simply use the extracted science rules directly as MLN clauses and exploit the structure present in hard constraints to improve tractability. Second, we interpret science rules as describing prototypical entities, resulting in a drastically simplified but brittle network. Our third approach, called Praline, uses MLNs to align lexical elements as well as define and control how inference should be performed in this task. Praline demonstrates a 15\% accuracy boost and a 10x reduction in runtime as compared to other MLN-based methods, and comparable accuracy to word-based baseline approaches.
