The prevalence of temporal references accross all types of natural language utterances makes temporal analysis a key issue in Natural Language Processing. This work adresses three research questions: 1/is temporal expression recognition specific to a particular domain? 2/if so, can we characterize domain specificity? and 3/how can sudomain  specificity be integrated in a single tool for unified temporal expression extraction? Herein, we assess temporal expression recognition from documents written in French covering three domains. We present a new corpus of clinical narratives annotated for temporal expressions, and also use existing corpora in the newswire and historical domains. We show that temporal expressions can be extracted with high performance across domains (best F-measure 0.96 obtained with a CRF model on clinical narratives). We argue that domain adaptation for the extraction of temporal expressions can be done with limited efforts and should cover pre-processing as well as temporal specific tasks.
