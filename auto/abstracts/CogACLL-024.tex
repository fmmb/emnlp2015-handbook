This study investigates the use of syllables and phone(me)s in computational models of segmentation in early language acquisition. We results of experiments with both syllables and phonemes as the basic unit using a standard state-of-the-art segmentation model. We evaluate the model output based on both word- and morpheme-segmented gold standards on child-directed speech corpora from two typologically different languages. Our results do not indicate a clear advantage for one unit or the other. We argue that the computational advantage for the syllable suggested in earlier research may be an artifact of the particular language and/or segmentation strategy used in these studies.
